\documentclass[12pt,oneside,english]{book}
\usepackage{babel}
\usepackage[utf8]{inputenc}
\usepackage[T1]{fontenc}
\usepackage{color}
\definecolor{marron}{RGB}{60,30,10}
\definecolor{darkblue}{RGB}{0,0,80}
\definecolor{lightblue}{RGB}{80,80,80}
\definecolor{darkgreen}{RGB}{0,80,0}
\definecolor{darkgray}{RGB}{0,80,0}
\definecolor{darkred}{RGB}{80,0,0}
\definecolor{shadecolor}{rgb}{0.97,0.97,0.97}
\usepackage{graphicx}
\usepackage{wallpaper}
\usepackage{wrapfig,booktabs}

\usepackage{fancyhdr}
\usepackage{lettrine}
\input Acorn.fd

\renewcommand{\familydefault}{pplj} 
\usepackage[
final,
stretch=10,
protrusion=true,
tracking=true,
spacing=on,
kerning=on,
expansion=true]{microtype}

%\setlength{\parskip}{1.5ex plus 0.2ex minus 0.2ex}


\usepackage{fourier-orns}
\newcommand{\dash}{\vspace{2em}\noindent \textcolor{darkgray}{\hrulefill~ \raisebox{-2.5pt}[10pt][10pt]{\leafright \decofourleft \decothreeleft  \aldineright \decotwo \floweroneleft \decoone   \floweroneright \decotwo \aldineleft\decothreeright \decofourright \leafleft} ~  \hrulefill \\ \vspace{2em}}}
\newcommand{\rdash}{\noindent \textcolor{darkgray}{ \raisebox{-1.9pt}[10pt][10pt]{\leafright} \hrulefill \raisebox{-1.9pt}[10pt][10pt]{\leafright \decofourleft \decothreeleft  \aldineright \decotwo \floweroneleft \decoone}}}
\newcommand{\ldash}{\textcolor{darkgray}{\raisebox{-1.9pt}[10pt][10pt]{\decoone \floweroneright \decotwo \aldineleft \decothreeright \decofourright \leafleft} \hrulefill \raisebox{-1.9pt}[10pt][10pt]{\leafleft}}}

\fancyhf{}

\renewcommand{\chaptermark}[1]{\markboth{#1}{}}
\renewcommand{\sectionmark}[1]{\markright{#1}}

\newcommand{\estcab}[1]{\itshape\textcolor{marron}{\nouppercase #1}}

\fancyhead[LO]{\estcab{\rightmark}} 
\fancyhead[RO]{\estcab{\leftmark}}
\fancyhead[RO]{\bf\nouppercase{ \leftmark}}
\fancyfoot[RO]{ \leafNE  ~~ \bf \thepage}

\newenvironment{Section}[1]
{\section{\vspace{0ex}#1}}
{\vspace{12pt}\centering ------- \decofourleft\decofourright ------- \par}

\usepackage{lipsum}
\setlength{\parindent}{1em} % Sangría española
\pagestyle{fancy}

\renewcommand{\footnoterule}{\noindent\textcolor{marron}{\decosix \raisebox{2.9pt}{\line(1,0){100}} \lefthand} \vspace{.5em} }
\usepackage[hang,splitrule]{footmisc}
\addtolength{\footskip}{0.5cm}
\setlength{\footnotemargin}{0.3cm}
\setlength{\footnotesep}{0.4cm} 

\usepackage{chngcntr}
\counterwithout{figure}{chapter}
\counterwithout{table}{chapter}

\usepackage{kotex}
\usepackage{amsthm} 
\usepackage{amsmath} 
\usepackage{amsfonts}
\usepackage{enumerate} 
\usepackage{cite}
\usepackage{graphics} 
\usepackage{graphicx,lscape} 
\usepackage{subcaption}
\usepackage{algpseudocode}
\usepackage{algorithm}
\usepackage{titlesec}
\usepackage{cite, url}
\usepackage{amssymb}
\usepackage{xcolor}

\def\ck{\paragraph{\Large$\bullet$}\Large}
\def\sol{\paragraph{\Large(sol)}\Large}
\def\pf{\paragraph{\Large(pf)}\Large}
\def\note{\paragraph{\Large\textit{\underline{note:}}}\Large}
\def\ex{\paragraph{\Large\textit{example:}}\Large}
\newcommand{\para}[1]{\paragraph{\Large(#1)}\Large}
\newcommand{\parablue}[1]{\paragraph{\Large\textcolor{blue}{(#1)}}\Large}
\newcommand{\parared}[1]{\paragraph{\Large\textcolor{red}{(#1)}}\Large}

\def\one{\paragraph{\Large(1)}\Large}
\def\two{\paragraph{\Large(2)}\Large}
\def\three{\paragraph{\Large(3)}\Large}
\def\four{\paragraph{\Large(4)}\Large}
\def\five{\paragraph{\Large(5)}\Large}
\def\six{\paragraph{\Large(6)}\Large}
\def\seven{\paragraph{\Large(7)}\Large}
\def\eight{\paragraph{\Large(8)}\Large}
\def\nine{\paragraph{\Large(9)}\Large}
\def\ten{\paragraph{\Large(10)}\Large}

\def\cka{\paragraph{\Large(a)}\Large}
\def\ckb{\paragraph{\Large(b)}\Large}
\def\ckc{\paragraph{\Large(c)}\Large}
\def\ckd{\paragraph{\Large(d)}\Large}
\def\cke{\paragraph{\Large(e)}\Large}
\def\ckf{\paragraph{\Large(f)}\Large}
\def\ckg{\paragraph{\Large(g)}\Large}
\def\ckh{\paragraph{\Large(h)}\Large}
\def\cki{\paragraph{\Large(i)}\Large}
\def\ckj{\paragraph{\Large(j)}\Large}

\newcommand{\bld}[1]{\mbox{\boldmath $#1$}}
\newcommand{\bs}[1]{\mbox{\boldmath $#1$}}

\newcommand{\bsa}{\mbox{\boldmath $a$}}
\newcommand{\bsb}{\mbox{\boldmath $b$}}
\newcommand{\bsc}{\mbox{\boldmath $c$}}
\newcommand{\bsd}{\mbox{\boldmath $d$}}
\newcommand{\bse}{\mbox{\boldmath $e$}}
\newcommand{\bsf}{\mbox{\boldmath $f$}}
\newcommand{\bsg}{\mbox{\boldmath $g$}}
\newcommand{\bsh}{\mbox{\boldmath $h$}}
\newcommand{\bsi}{\mbox{\boldmath $i$}}
\newcommand{\bsj}{\mbox{\boldmath $j$}}
\newcommand{\bsk}{\mbox{\boldmath $k$}}
\newcommand{\bsl}{\mbox{\boldmath $l$}}
\newcommand{\bsm}{\mbox{\boldmath $m$}}
\newcommand{\bsn}{\mbox{\boldmath $n$}}
\newcommand{\bso}{\mbox{\boldmath $o$}}
\newcommand{\bsp}{\mbox{\boldmath $p$}}
\newcommand{\bsq}{\mbox{\boldmath $q$}}
\newcommand{\bsr}{\mbox{\boldmath $r$}}
\newcommand{\bss}{\mbox{\boldmath $s$}}
\newcommand{\bst}{\mbox{\boldmath $t$}}
\newcommand{\bsu}{\mbox{\boldmath $u$}}
\newcommand{\bsv}{\mbox{\boldmath $v$}}
\newcommand{\bsw}{\mbox{\boldmath $w$}}
\newcommand{\bsx}{\mbox{\boldmath $x$}}
\newcommand{\bsy}{\mbox{\boldmath $y$}}
\newcommand{\bsz}{\mbox{\boldmath $z$}}

\newcommand{\bfa}{\mbox{$\bf{a}$}}
\newcommand{\bfb}{\mbox{$\bf{b}$}}
\newcommand{\bfc}{\mbox{$\bf{c}$}}
\newcommand{\bfd}{\mbox{$\bf{d}$}}
\newcommand{\bfe}{\mbox{$\bf{e}$}}
\newcommand{\bff}{\mbox{$\bf{f}$}}
\newcommand{\bfg}{\mbox{$\bf{g}$}}
\newcommand{\bfh}{\mbox{$\bf{h}$}}
\newcommand{\bfi}{\mbox{$\bf{i}$}}
\newcommand{\bfj}{\mbox{$\bf{j}$}}
\newcommand{\bfk}{\mbox{$\bf{k}$}}
\newcommand{\bfl}{\mbox{$\bf{l}$}}
\newcommand{\bfm}{\mbox{$\bf{m}$}}
\newcommand{\bfn}{\mbox{$\bf{n}$}}
\newcommand{\bfo}{\mbox{$\bf{o}$}}
\newcommand{\bfp}{\mbox{$\bf{p}$}}
\newcommand{\bfq}{\mbox{$\bf{q}$}}
\newcommand{\bfr}{\mbox{$\bf{r}$}}
\newcommand{\bfs}{\mbox{$\bf{s}$}}
\newcommand{\bft}{\mbox{$\bf{t}$}}
\newcommand{\bfu}{\mbox{$\bf{u}$}}
\newcommand{\bfv}{\mbox{$\bf{v}$}}
\newcommand{\bfw}{\mbox{$\bf{w}$}}
\newcommand{\bfx}{\mbox{$\bf{x}$}}
\newcommand{\bfy}{\mbox{$\bf{y}$}}
\newcommand{\bfz}{\mbox{$\bf{z}$}}

\newcommand{\bsA}{\mbox{$\boldmath{A}$}}
\newcommand{\bsB}{\mbox{$\boldmath{B}$}}
\newcommand{\bsC}{\mbox{$\boldmath{C}$}}
\newcommand{\bsD}{\mbox{$\boldmath{D}$}}
\newcommand{\bsE}{\mbox{$\boldmath{E}$}}
\newcommand{\bsF}{\mbox{$\boldmath{F}$}}
\newcommand{\bsG}{\mbox{$\boldmath{G}$}}
\newcommand{\bsH}{\mbox{$\boldmath{H}$}}
\newcommand{\bsI}{\mbox{$\boldmath{I}$}}
\newcommand{\bsJ}{\mbox{$\boldmath{J}$}}
\newcommand{\bsK}{\mbox{$\boldmath{K}$}}
\newcommand{\bsL}{\mbox{$\boldmath{L}$}}
\newcommand{\bsM}{\mbox{$\boldmath{M}$}}
\newcommand{\bsN}{\mbox{$\boldmath{N}$}}
\newcommand{\bsO}{\mbox{$\boldmath{O}$}}
\newcommand{\bsP}{\mbox{$\boldmath{P}$}}
\newcommand{\bsQ}{\mbox{$\boldmath{Q}$}}
\newcommand{\bsR}{\mbox{$\boldmath{R}$}}
\newcommand{\bsS}{\mbox{$\boldmath{S}$}}
\newcommand{\bsT}{\mbox{$\boldmath{T}$}}
\newcommand{\bsU}{\mbox{$\boldmath{U}$}}
\newcommand{\bsV}{\mbox{$\boldmath{V}$}}
\newcommand{\bsW}{\mbox{$\boldmath{W}$}}
\newcommand{\bsX}{\mbox{$\boldmath{X}$}}
\newcommand{\bsY}{\mbox{$\boldmath{Y}$}}
\newcommand{\bsZ}{\mbox{$\boldmath{Z}$}}

\newcommand{\bfA}{\mbox{$\bf{A}$}}
\newcommand{\bfB}{\mbox{$\bf{B}$}}
\newcommand{\bfC}{\mbox{$\bf{C}$}}
\newcommand{\bfD}{\mbox{$\bf{D}$}}
\newcommand{\bfE}{\mbox{$\bf{E}$}}
\newcommand{\bfF}{\mbox{$\bf{F}$}}
\newcommand{\bfG}{\mbox{$\bf{G}$}}
\newcommand{\bfH}{\mbox{$\bf{H}$}}
\newcommand{\bfI}{\mbox{$\bf{I}$}}
\newcommand{\bfJ}{\mbox{$\bf{J}$}}
\newcommand{\bfK}{\mbox{$\bf{K}$}}
\newcommand{\bfL}{\mbox{$\bf{L}$}}
\newcommand{\bfM}{\mbox{$\bf{M}$}}
\newcommand{\bfN}{\mbox{$\bf{N}$}}
\newcommand{\bfO}{\mbox{$\bf{O}$}}
\newcommand{\bfP}{\mbox{$\bf{P}$}}
\newcommand{\bfQ}{\mbox{$\bf{Q}$}}
\newcommand{\bfR}{\mbox{$\bf{R}$}}
\newcommand{\bfS}{\mbox{$\bf{S}$}}
\newcommand{\bfT}{\mbox{$\bf{T}$}}
\newcommand{\bfU}{\mbox{$\bf{U}$}}
\newcommand{\bfV}{\mbox{$\bf{V}$}}
\newcommand{\bfW}{\mbox{$\bf{W}$}}
\newcommand{\bfX}{\mbox{$\bf{X}$}}
\newcommand{\bfY}{\mbox{$\bf{Y}$}}
\newcommand{\bfZ}{\mbox{$\bf{Z}$}}

\DeclareMathOperator*{\argmin}{argmin} %\usepackage{amsmath}를 써야지 정의가능함. 
\DeclareMathOperator*{\argmax}{argmax} %\usepackage{amsmath}를 써야지 정의가능함. 

\usepackage{titlesec}
\titleformat*{\section}{\huge\bfseries}
\titleformat*{\subsection}{\huge\bfseries}
\titleformat*{\subsubsection}{\huge\bfseries}
\titleformat*{\paragraph}{\huge\bfseries}
\titleformat*{\subparagraph}{\huge\bfseries}


\titleclass{\part}{top}
\titleformat{\part}[display]
  {\normalfont\huge\bfseries}{\centering\partname\ \thepart}{20pt}{\Huge\centering}
\titlespacing*{\part}{0pt}{50pt}{40pt}
\titleclass{\chapter}{straight}
\titleformat{\chapter}[display]
  {\normalfont\huge\bfseries}{\chaptertitlename\ \thechapter}{20pt}{\Huge}
\titlespacing*{\chapter} {0pt}{50pt}{40pt}

\newcommand*\initfamily{\usefont{U}{Acorn}{xl}{n}}
\usepackage[left=10px,right=10px,top=10px,bottom=10px,paperwidth=8in,paperheight=15in]{geometry}

\usepackage{geometry}
\geometry{
tmargin=3cm, 
bmargin=3cm, 
lmargin=1cm, 
rmargin=1cm,
headheight=1.5cm,
headsep=0.8cm,
footskip=0.5cm}
\linespread{1}
\begin{document}

\subsubsection{1. Definitions}
\parablue{def} $(S,{\cal S})$를 measurable space라고 하자. $X_n:(\Omega,{\cal F}) \to (S,{\cal S})$이라고 하자. 편하게 
\[
(S,{\cal S})=(\mathbb{R},{\cal R})
\]
이라고 생각해도 무방하다. 어떠한 확률변수열 $\{X_n\}$이 filtration ${\cal F}_n:=\sigma(X_0,\dots,X_n)$에서 정의되어 있다고 하자. 확률변수열 $\{X_n\}$이 마코프체인이라는 것은 아래와 같이 정의한다. 
\begin{align*}
& \{X_n\} \mbox{ is Markovchain w.r.t. } {\cal F}_n \\
& \overset{def}{\Longleftrightarrow} \mbox{for all $B\in {\cal S}$:}\quad 
P(X_{n+1}\in B | {\cal F}_n)=P(X_{n+1}\in B | X_n)
\end{align*}

\note 확률변수열 $X_1,X_2,\dots$의 값이 바로 이전의 값에 의해서만 결정되면 마코프체인이라고 한다. 즉 $X_2$의 값을 알기 위해서는 $X_1$의 값에 대한 정보만 있으면 되고 $X_3$의 값을 알기 위해서는 $X_2$에 대한 정보만 있으면 될때 $X_1,X_2,\dots$을 마코프체인이라고 한다. 

\ck $4\times 4$ 그리드 세계를 가정하자. 
\[
\Omega=\{(1,1),\dots,(4,4)\}
\]
이고 
\[
S=\{1,2,3\dots,16\}
\]
이라고 하자. 확률변수 $X_1:(\Omega,{\cal F}) \to (S, {\cal S})$은 아래와 같이 정의할 수 있는 맵핑이라고 하자. 
\begin{align*}
& X_1\big((1,1)\big)=1\\ 
& X_2\big((1,2))=2\\ 
& \dots \\
& X_1\big((4,4)\big)=16
\end{align*}
따라서 
\[
X_1=1, X_2=2
\]
가 의미하는 것은 처음에는 $(1,1)$의 위치에 있다가 그다음에는 $(1,2)$의 위치로 이동하였다는 것을 의미한다. 이제 $(1,1)$의 위치에서 $(1,2)$의 위치로 이동하는 transition probability를 $p$라고 정의하자. 여기에서 $p$는 확률공간을 구성하는 $P$이지 pmf를 의미하는 것이 아님을 기억하자. $p$는 아래와 같이 정의할 수 있다. 
\[
p:(S,{\cal S}) \to \mathbb{R}
\]
기호로는 아래와 같이 쓴다. 
\[
p(x,A)
\]
여기에서 $x \in S$, $A \in {\cal S}$이다. 

\ex $x=1$, $A=\{1,2,5\}$이라고 하자. 
\begin{align*}
& x=1 \Leftrightarrow X(\omega)=1 \Leftrightarrow \omega=(1,1) \\ 
& A=\{1,2,5\} \Leftrightarrow \{\omega:X(\omega) \in A\}=\{(1,1),(1,2),(2,1)\}
\end{align*}
임을 주목하라. 따라서 
\[
p(x,A)
\] 
는 점 $(1,1)$에서 출발했는데 점 $(1,1),(1,2),(2,1)$중 하나에 도착할 확률이므로 
\[
p(x,A)=1
\]
이라고 볼 수 있다. 

\parablue{def} transition probability의 정의를 사용하면 아래를 만족하는 확률변수열 $\{X_n\}$을 마코프체인이라 정의할 수 있다. 
\[
P(X_{n+1}\in B|{\cal F}_n)=p_n(X_n,B)
\]
여기에서 $p_n$은 $n$번째에 어떠한 위치 $X_n$에서 $B$의 부분집합중 하나의 위치로 이동할 확률을 의미한다. 

\parared{결론1,2,3의 가정} 만약에 (1) $(S,{\cal S})$이 \textcolor{red}{\emph{nice space}} 이고 (2) $\{p_n\}$이 잘 정의되며 (3) $(S,{\cal S})$에서의 initional distribution $\mu$가 잘 정의된다고 하자. 

\parared{결론1} 일단 유한개의 확률변수열 $\{X_n\}$에 대하여 consistence set of finite dimensional distribution을 아래와 같이 잘 정의할 수 있다. 
\begin{align*}
& \mbox{Prob}(X_1\in B_1, X_2\in B_2, \dots, X_n \in B_n)\\
& = P(X_1\in B_1, X_2\in B_2, \dots, X_n \in B_n)\\
& = \int_{B_0}\mu(dx_0)\int_{B_1}p_0\big(x_0,\mu(dx_1)\big)\dots\int_{B_n}p_{n-1}\big(x_{n-1},\mu(dx_n)\big)
\end{align*}
\note 이때 $\mbox{Prob}(X_1\in B_1, X_2\in B_2, \dots, X_n \in B_n)$와 같은 표현은 이해하기 쉽지만 수학적으로 엄밀하지 않은 표현이다. 따라서 엄밀하게 하려면 아래의 공간에서 정의되는 확률측도 $P$를 사용하여 표현해야한다. 
\[
(S_0\times S_1 \dots \times S_n, {\cal S}_0\times {\cal S}_1 \dots \times {\cal S}_n)
\]
이 공간은 간단하게 아래와 같이 표현하기도 한다. 
\[
\left(S^{\{0,1,\dots,n\}},{\cal S}^{\{0,1,\dots,n\}}\right)
\]

\note 즉 결론1은 $\left(S^{\{0,1,\dots,n\}},{\cal S}^{\{0,1,\dots,n\}}\right)$에서의 확률측도 $P$는 (혹은 임의의 유한 확률변수열 $\{X_n\}$에 대한 확률측도 $P$는) 초기분포 $\mu$와 $p_n$만 잘 정의하면 모순없이 정의가능하다는 것을 의미한다.

\ck 하지만 무한일 경우에도 잘 정의될까? 

\parared{결론2} (1)-(3)의 가정하에 Kolmogorov's theorem은 확률변수열 $\{X_n\}$이 무한수열을 가지더라도 아래와 같은 확률이 잘 정의됨을 보여준다. 
\[
\mbox{Prob}(X_1\in B_1, X_2\in B_2, \dots, )
\]
즉 이는 $\mu$와 $\{p_n\}$만 잘 정의되면 위의 같은 확률들을 모순없이 정의할 수 있음을 의미한다. 교재에서는 유한인 경우와 구분하기 위해서 위의 확률을 표현하는 확률측도를 특별히 $P_{\mu}$라고 하였다. 즉 아래와 같이 써야 올바르다. 
\[
P_{\mu}(X_1\in B_1, X_2\in B_2, \dots)
\]
이때 $P_{\mu}$는 $\left(S^{\{0,1,\dots\}},{\cal S}^{\{0,1,\dots,\}}\right)$에서의 확률측도이다. 

\note 아래의 기호는 외우는 것이 좋겠다. 
\begin{align*}
& \mbox{Prob}(X_0\in B_0)= \int_{B_0} \mu(dx_0)\\ 
& \mbox{Prob}(X_0\in B_0,X_1 \in B_1)= \int_{B_0} \mu(d(x_0)) \int_{B_1} p\big(x_0,\mu(dx_1)\big) \\ 
\end{align*}

\ck 지금까지는 콜모고로프의 정리덕에 $P_{\mu}$가 잘 정의된다는 사실까지 살펴보았다. 즉 (1) 초기분포 $\mu$와 (2) transition 확률 $\{p_n\}$이 잘 정의되면 무한하게 눈을 쌓아도 $P_{\mu}$가 잘 정의된다. 

\parared{결론3, Thm 6.1.1} (1)-(3)의 조건하에 $\{X_n\}$이 마코프체인이 된다. 
\pf 아래의 기호를 정의하면서 증명을 시작하자. 
\parablue{notation}
$\mu=\delta_x$를 $x$에서의 포인트매스라고 하자. 그리고 기호 $P_x=P_{\delta_x}$라고 정의하자. $P_x$가 정의되면 아래와 같이 $P_{\mu}$를 정의할 수 있다. 
\[
P_{\mu}(A)=\int \mu(dx)P_x(A), \quad A \in {\cal S}^{\{0,1,\dots,\}}
\]

\parared{Thm 6.1.2, 결론1의 변형} (1)-(3)의 조건중 체크하기 까다로운 것은 (1)이다. 오히려 (1)의 조건대신에 $\{X_n\}$이 마코프체인임을 가정하면 결론2와 동일한 결과를 얻을 수 있다. 즉 (1) $\{X_n\}$이 마코프체인이고 (2) transition prob $\{p_n\}$이 주어졌고 (3) initional distribution $\mu$ 가 주어졌다면 \textcolor{red}{\emph{finite dimensional distribution}}이 아래와 같이 주어진다. 
\begin{align*}
& P(X_j \in B_j, 0\leq j \leq n) \\ 
& = P(X_1\in B_1, X_2\in B_2, \dots, X_n \in B_n)\\
& = \int_{B_0}\mu(dx_0)\int_{B_1}p_0\big(x_0,\mu(dx_1)\big)\dots\int_{B_n}p_{n-1}\big(x_{n-1},\mu(dx_n)\big)
\end{align*}

\subsubsection{Examples}

\subsubsection{Extensions of the Markov Property}

\subsubsection{Recurrence and Transience}

\subsubsection{Stationary Measures}
\ck 아래식을 만족하는 measure $\mu$를 stationary measure라고 한다. 
\[
\sum_x\mu(x)p(x,y)=\mu(y)
\]
\note $p(x,y)$: 노드 $x$에서 다음 노드 $y$로 이동할 확률
\note $\mu(x)$: 노드 $x$에 있을 확률
\note 따라서 stationary measure는 특정노드에 있을 확률을 측정하는 메져라 생각할 수 있다. 


\ck stationary measure(=stationary distribution)가 (1) 존재하고 (2) 유일하다는 것이 조사되었다고 하자. 이제 다음 관심사는 아래식을 만족하는 staionary distribution $\pi$이다. 
\[
\pi p =\pi
\]

\parablue{정리 6.5.6.} 

\parablue{정리 6.5.6.} $p$가 irreducible 하다는 것과 아래는 동치이다. 
\one .
\two stationary distribution이 존재한다. 
\three . 


\subsubsection{Asymptotic Behavior}
\parablue{레마 6.6.3.} $d_x=1$이라면 $m_0$보다 큰 모든 $m$에 대하여 
\[
p^m(x,x)>0
\]
를 만족시킬 수 있다. 

\parablue{정리 6.6.4.} $p$가 (1) irreducible 하고 (2) aperiodic 하며 (3) stationary distribution $\pi$를 가진다고 하자. 그러면 아래가 성립한다. 
\[
p^{n}(x,y) \to \pi(y) \quad as~ n\to \infty .
\]
\note $p$가 irreducible 인것만 보이면 stationary distribution $\pi$를 가진다는 것은 정리 6.5.6에 의해서 성립한다. 따라서 (1)-(2)만 조건으로 사용해도 위의 정리는 성립한다. 
\note $p$가 에이피리오딕하다는 의미는 모든 state가 $d_x=1$을 가진다는 것을 의미한다 .

\pf

\ck $S^2=S\times S$ 라고 하자. 
\ck 전이확률 $\bar{p}$를 $S\times S$에서 아래와 같이 정의하자. 
\[
\bar{p}\big((x_1,y_1),(x_2,y_2)\big)=p(x_1,x_2)p(y_1,y_2)
\]
\note 이는 각각의 coordinate가 독립적으로 움직인다는 것을 의미한다. 
\parared{step 1} 
\ck 먼저 $\bar{p}$가 이리듀시블임을 보이자. (이는 너무 당연해서 바보같은 증명으로 보이지만 정리의 에이피리오딕조건을 사용하는 유일한 과정이다.) 우선 $p$가 이리듀시블하다는 조건으로부터 아래를 만족하는 적당한 $K,L$이 존재함을 알 수 있다. 
\[
p^K(x_1,x_2)>0 \quad \mbox {and}\quad p^L(y_1,y_2)>0.
\]
그런데 레마 6.6.3에 의해서 $M$을 적당히 크게 설정한다면 아래를 만족시킬수 있음을 알 수 있다. 
\[
p^{L+M}(x_1,x_2)>0 \quad \mbox {and}\quad p^{K+M}(y_1,y_2)>0.
\]
따라서 아래가 성립한다. 
\[
\bar{p}^{K+L+M}\big((x_1,y_1),(x_2,y_2) \big)>0
\]

\parared{step 2} 
\ck 두 코디네이츠가 독립이므로 ${\bar p}$의 stationary distribution을 아래와 같이 정의할 수 있다. 
\[
\bar{\pi}(a,b)=\pi(a)\pi(b).
\]

\ck 정리 6.5.4에 의해서 $\bar{p}$의 stationary distribution이 존재한다는 것은 $\bar{p}$의 모든상태가 recurrent하다는 것을 의미한다. 

\ck $(X_n,Y_n)$을 $S \times S$에서의 체인이라고 하자. 

\ck $T$를 이 체인이 처음으로 대각 $\{(y,y)\in S\}$을 치는 시간이라고 하자. 

\ck $T(x,x)$를 $(x,x)$를 hit하는 시간이라고하자. 

\ck $\bar{p}$가 (1) irreducible 하고 (2) recurrent 하므로 $T(x,x)<\infty ~ a.s.$ 이고 따라서 $T <\infty ~ a.s.$ 이다. 

\parared{step 3} 

\ck 우선 두개의 코디네이트 $(X_n,Y_n)$가 $\{T\leq n\}$에서 같은 분포를 가진다는 것을 관찰하자. 

\ck $(X_n,Y_n)$이 첫 교차점을 가지는 시간과 장소를 고려하여보자. 마코프성질을 이용하면 
\[
P(X_n=y,T\leq n)= \sum_{m=1}^{n}\sum_xP(T=m,X_m=x,X_n=y)
\]


\subsubsection{Periodicity, Tail $\sigma$-field}

\subsubsection{General State}


\end{document}

