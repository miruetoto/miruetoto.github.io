\documentclass[12pt,oneside,english]{book}
\usepackage{babel}
\usepackage[utf8]{inputenc}
\usepackage[T1]{fontenc}
\usepackage{color}
\definecolor{marron}{RGB}{60,30,10}
\definecolor{darkblue}{RGB}{0,0,80}
\definecolor{lightblue}{RGB}{80,80,80}
\definecolor{darkgreen}{RGB}{0,80,0}
\definecolor{darkgray}{RGB}{0,80,0}
\definecolor{darkred}{RGB}{80,0,0}
\definecolor{shadecolor}{rgb}{0.97,0.97,0.97}
\usepackage{graphicx}
\usepackage{wallpaper}
\usepackage{wrapfig,booktabs}

\usepackage{fancyhdr}
\usepackage{lettrine}
\input Acorn.fd

\renewcommand{\familydefault}{pplj} 
\usepackage[
final,
stretch=10,
protrusion=true,
tracking=true,
spacing=on,
kerning=on,
expansion=true]{microtype}

%\setlength{\parskip}{1.5ex plus 0.2ex minus 0.2ex}


\usepackage{fourier-orns}
\newcommand{\dash}{\vspace{2em}\noindent \textcolor{darkgray}{\hrulefill~ \raisebox{-2.5pt}[10pt][10pt]{\leafright \decofourleft \decothreeleft  \aldineright \decotwo \floweroneleft \decoone   \floweroneright \decotwo \aldineleft\decothreeright \decofourright \leafleft} ~  \hrulefill \\ \vspace{2em}}}
\newcommand{\rdash}{\noindent \textcolor{darkgray}{ \raisebox{-1.9pt}[10pt][10pt]{\leafright} \hrulefill \raisebox{-1.9pt}[10pt][10pt]{\leafright \decofourleft \decothreeleft  \aldineright \decotwo \floweroneleft \decoone}}}
\newcommand{\ldash}{\textcolor{darkgray}{\raisebox{-1.9pt}[10pt][10pt]{\decoone \floweroneright \decotwo \aldineleft \decothreeright \decofourright \leafleft} \hrulefill \raisebox{-1.9pt}[10pt][10pt]{\leafleft}}}

\fancyhf{}

\renewcommand{\chaptermark}[1]{\markboth{#1}{}}
\renewcommand{\sectionmark}[1]{\markright{#1}}

\newcommand{\estcab}[1]{\itshape\textcolor{marron}{\nouppercase #1}}

\fancyhead[LO]{\estcab{\rightmark}} 
\fancyhead[RO]{\estcab{\leftmark}}
\fancyhead[RO]{\bf\nouppercase{ \leftmark}}
\fancyfoot[RO]{ \leafNE  ~~ \bf \thepage}

\newenvironment{Section}[1]
{\section{\vspace{0ex}#1}}
{\vspace{12pt}\centering ------- \decofourleft\decofourright ------- \par}

\usepackage{lipsum}
\setlength{\parindent}{1em} % Sangría española
\pagestyle{fancy}

\renewcommand{\footnoterule}{\noindent\textcolor{marron}{\decosix \raisebox{2.9pt}{\line(1,0){100}} \lefthand} \vspace{.5em} }
\usepackage[hang,splitrule]{footmisc}
\addtolength{\footskip}{0.5cm}
\setlength{\footnotemargin}{0.3cm}
\setlength{\footnotesep}{0.4cm} 

\usepackage{chngcntr}
\counterwithout{figure}{chapter}
\counterwithout{table}{chapter}

\usepackage{kotex}
\usepackage{amsthm} 
\usepackage{amsmath} 
\usepackage{amsfonts}
\usepackage{enumerate} 
\usepackage{cite}
\usepackage{graphics} 
\usepackage{graphicx,lscape} 
\usepackage{subcaption}
\usepackage{algpseudocode}
\usepackage{algorithm}
\usepackage{titlesec}
\usepackage{cite, url}
\usepackage{amssymb}

\def\ck{\paragraph{\Large$\bullet$}\Large}
\def\sol{\paragraph{\Large(sol)}\Large}
\def\pf{\paragraph{\Large(pf)}\Large}
\def\note{\paragraph{\Large\textit{\underline{note:}}}\Large}
\def\ex{\paragraph{\Large\textit{example:}}\Large}
\newcommand{\para}[1]{\paragraph{\Large(#1)}\Large}
\newcommand{\parablue}[1]{\paragraph{\Large\textcolor{blue}{(#1)}}\Large}
\newcommand{\parared}[1]{\paragraph{\Large\textcolor{red}{(#1)}}\Large}

\def\one{\paragraph{\Large(1)}\Large}
\def\two{\paragraph{\Large(2)}\Large}
\def\three{\paragraph{\Large(3)}\Large}
\def\four{\paragraph{\Large(4)}\Large}
\def\five{\paragraph{\Large(5)}\Large}
\def\six{\paragraph{\Large(6)}\Large}
\def\seven{\paragraph{\Large(7)}\Large}
\def\eight{\paragraph{\Large(8)}\Large}
\def\nine{\paragraph{\Large(9)}\Large}
\def\ten{\paragraph{\Large(10)}\Large}

\def\cka{\paragraph{\Large(a)}\Large}
\def\ckb{\paragraph{\Large(b)}\Large}
\def\ckc{\paragraph{\Large(c)}\Large}
\def\ckd{\paragraph{\Large(d)}\Large}
\def\cke{\paragraph{\Large(e)}\Large}
\def\ckf{\paragraph{\Large(f)}\Large}
\def\ckg{\paragraph{\Large(g)}\Large}
\def\ckh{\paragraph{\Large(h)}\Large}
\def\cki{\paragraph{\Large(i)}\Large}
\def\ckj{\paragraph{\Large(j)}\Large}

\newcommand{\bld}[1]{\mbox{\boldsymbol $#1$}}
\newcommand{\bs}[1]{\mbox{\boldsymbol $#1$}}

\newcommand{\bsa}{\mbox{\boldsymbol $a$}}
\newcommand{\bsb}{\mbox{\boldsymbol $b$}}
\newcommand{\bsc}{\mbox{\boldsymbol $c$}}
\newcommand{\bsd}{\mbox{\boldsymbol $d$}}
\newcommand{\bse}{\mbox{\boldsymbol $e$}}
\newcommand{\bsf}{\mbox{\boldsymbol $f$}}
\newcommand{\bsg}{\mbox{\boldsymbol $g$}}
\newcommand{\bsh}{\mbox{\boldsymbol $h$}}
\newcommand{\bsi}{\mbox{\boldsymbol $i$}}
\newcommand{\bsj}{\mbox{\boldsymbol $j$}}
\newcommand{\bsk}{\mbox{\boldsymbol $k$}}
\newcommand{\bsl}{\mbox{\boldsymbol $l$}}
\newcommand{\bsm}{\mbox{\boldsymbol $m$}}
\newcommand{\bsn}{\mbox{\boldsymbol $n$}}
\newcommand{\bso}{\mbox{\boldsymbol $o$}}
\newcommand{\bsp}{\mbox{\boldsymbol $p$}}
\newcommand{\bsq}{\mbox{\boldsymbol $q$}}
\newcommand{\bsr}{\mbox{\boldsymbol $r$}}
\newcommand{\bss}{\mbox{\boldsymbol $s$}}
\newcommand{\bst}{\mbox{\boldsymbol $t$}}
\newcommand{\bsu}{\mbox{\boldsymbol $u$}}
\newcommand{\bsv}{\mbox{\boldsymbol $v$}}
\newcommand{\bsw}{\mbox{\boldsymbol $w$}}
\newcommand{\bsx}{\mbox{\boldsymbol $x$}}
\newcommand{\bsy}{\mbox{\boldsymbol $y$}}
\newcommand{\bsz}{\mbox{\boldsymbol $z$}}

\newcommand{\bfa}{\mbox{$\bf{a}$}}
\newcommand{\bfb}{\mbox{$\bf{b}$}}
\newcommand{\bfc}{\mbox{$\bf{c}$}}
\newcommand{\bfd}{\mbox{$\bf{d}$}}
\newcommand{\bfe}{\mbox{$\bf{e}$}}
\newcommand{\bff}{\mbox{$\bf{f}$}}
\newcommand{\bfg}{\mbox{$\bf{g}$}}
\newcommand{\bfh}{\mbox{$\bf{h}$}}
\newcommand{\bfi}{\mbox{$\bf{i}$}}
\newcommand{\bfj}{\mbox{$\bf{j}$}}
\newcommand{\bfk}{\mbox{$\bf{k}$}}
\newcommand{\bfl}{\mbox{$\bf{l}$}}
\newcommand{\bfm}{\mbox{$\bf{m}$}}
\newcommand{\bfn}{\mbox{$\bf{n}$}}
\newcommand{\bfo}{\mbox{$\bf{o}$}}
\newcommand{\bfp}{\mbox{$\bf{p}$}}
\newcommand{\bfq}{\mbox{$\bf{q}$}}
\newcommand{\bfr}{\mbox{$\bf{r}$}}
\newcommand{\bfs}{\mbox{$\bf{s}$}}
\newcommand{\bft}{\mbox{$\bf{t}$}}
\newcommand{\bfu}{\mbox{$\bf{u}$}}
\newcommand{\bfv}{\mbox{$\bf{v}$}}
\newcommand{\bfw}{\mbox{$\bf{w}$}}
\newcommand{\bfx}{\mbox{$\bf{x}$}}
\newcommand{\bfy}{\mbox{$\bf{y}$}}
\newcommand{\bfz}{\mbox{$\bf{z}$}}

\newcommand{\bsA}{\mbox{$\boldsymbol{A}$}}
\newcommand{\bsB}{\mbox{$\boldsymbol{B}$}}
\newcommand{\bsC}{\mbox{$\boldsymbol{C}$}}
\newcommand{\bsD}{\mbox{$\boldsymbol{D}$}}
\newcommand{\bsE}{\mbox{$\boldsymbol{E}$}}
\newcommand{\bsF}{\mbox{$\boldsymbol{F}$}}
\newcommand{\bsG}{\mbox{$\boldsymbol{G}$}}
\newcommand{\bsH}{\mbox{$\boldsymbol{H}$}}
\newcommand{\bsI}{\mbox{$\boldsymbol{I}$}}
\newcommand{\bsJ}{\mbox{$\boldsymbol{J}$}}
\newcommand{\bsK}{\mbox{$\boldsymbol{K}$}}
\newcommand{\bsL}{\mbox{$\boldsymbol{L}$}}
\newcommand{\bsM}{\mbox{$\boldsymbol{M}$}}
\newcommand{\bsN}{\mbox{$\boldsymbol{N}$}}
\newcommand{\bsO}{\mbox{$\boldsymbol{O}$}}
\newcommand{\bsP}{\mbox{$\boldsymbol{P}$}}
\newcommand{\bsQ}{\mbox{$\boldsymbol{Q}$}}
\newcommand{\bsR}{\mbox{$\boldsymbol{R}$}}
\newcommand{\bsS}{\mbox{$\boldsymbol{S}$}}
\newcommand{\bsT}{\mbox{$\boldsymbol{T}$}}
\newcommand{\bsU}{\mbox{$\boldsymbol{U}$}}
\newcommand{\bsV}{\mbox{$\boldsymbol{V}$}}
\newcommand{\bsW}{\mbox{$\boldsymbol{W}$}}
\newcommand{\bsX}{\mbox{$\boldsymbol{X}$}}
\newcommand{\bsY}{\mbox{$\boldsymbol{Y}$}}
\newcommand{\bsZ}{\mbox{$\boldsymbol{Z}$}}

\newcommand{\bfA}{\mbox{$\bf{A}$}}
\newcommand{\bfB}{\mbox{$\bf{B}$}}
\newcommand{\bfC}{\mbox{$\bf{C}$}}
\newcommand{\bfD}{\mbox{$\bf{D}$}}
\newcommand{\bfE}{\mbox{$\bf{E}$}}
\newcommand{\bfF}{\mbox{$\bf{F}$}}
\newcommand{\bfG}{\mbox{$\bf{G}$}}
\newcommand{\bfH}{\mbox{$\bf{H}$}}
\newcommand{\bfI}{\mbox{$\bf{I}$}}
\newcommand{\bfJ}{\mbox{$\bf{J}$}}
\newcommand{\bfK}{\mbox{$\bf{K}$}}
\newcommand{\bfL}{\mbox{$\bf{L}$}}
\newcommand{\bfM}{\mbox{$\bf{M}$}}
\newcommand{\bfN}{\mbox{$\bf{N}$}}
\newcommand{\bfO}{\mbox{$\bf{O}$}}
\newcommand{\bfP}{\mbox{$\bf{P}$}}
\newcommand{\bfQ}{\mbox{$\bf{Q}$}}
\newcommand{\bfR}{\mbox{$\bf{R}$}}
\newcommand{\bfS}{\mbox{$\bf{S}$}}
\newcommand{\bfT}{\mbox{$\bf{T}$}}
\newcommand{\bfU}{\mbox{$\bf{U}$}}
\newcommand{\bfV}{\mbox{$\bf{V}$}}
\newcommand{\bfW}{\mbox{$\bf{W}$}}
\newcommand{\bfX}{\mbox{$\bf{X}$}}
\newcommand{\bfY}{\mbox{$\bf{Y}$}}
\newcommand{\bfZ}{\mbox{$\bf{Z}$}}

\DeclareMathOperator*{\argmin}{argmin} %\usepackage{amsmath}를 써야지 정의가능함. 
\DeclareMathOperator*{\argmax}{argmax} %\usepackage{amsmath}를 써야지 정의가능함. 

\usepackage{titlesec}
\titleformat*{\section}{\huge\bfseries}
\titleformat*{\subsection}{\huge\bfseries}
\titleformat*{\subsubsection}{\huge\bfseries}
\titleformat*{\paragraph}{\huge\bfseries}
\titleformat*{\subparagraph}{\huge\bfseries}


\titleclass{\part}{top}
\titleformat{\part}[display]
  {\normalfont\huge\bfseries}{\centering\partname\ \thepart}{20pt}{\Huge\centering}
\titlespacing*{\part}{0pt}{50pt}{40pt}
\titleclass{\chapter}{straight}
\titleformat{\chapter}[display]
  {\normalfont\huge\bfseries}{\chaptertitlename\ \thechapter}{20pt}{\Huge}
\titlespacing*{\chapter} {0pt}{50pt}{40pt}

\newcommand*\initfamily{\usefont{U}{Acorn}{xl}{n}}
\usepackage[left=10px,right=10px,top=10px,bottom=10px,paperwidth=8in,paperheight=100in]{geometry}

\usepackage{geometry}
\geometry{
tmargin=3cm, 
bmargin=3cm, 
lmargin=1cm, 
rmargin=1cm,
headheight=1.5cm,
headsep=0.8cm,
footskip=0.5cm}
\linespread{1}
\begin{document}

\subsubsection{4.1. Stopping Times}
\ck 아래를 가정하자. 
\begin{align*}
& \Omega=\{(\omega_1,\omega_2,\dots,):\omega_i \in S \} \\ 
& {\cal F}= {\cal S} \times {\cal S} \times \dots \\ 
& P=\mu \times \mu \times  \dots  \quad \mbox{$\mu$ is the distribution of $X_i$.} \\ 
& X_n(\omega)=\omega_n
\end{align*}

\ck 맵핑 $\pi:\mathbb{N} \to \mathbb{N}$ 가 \emph{finite permutation}이라는 것은 유한개의 $i$에 대해서면 $\pi(i)\neq i$가 성립한다는 의미이다. 따라서 퍼퓨테이션은 인덱스의 순서를 (유한개) 바꿔주는 함수라 볼 수 있다. 

\ck 만약에 $\pi$가 $\mathbb{N}$의 \emph{finite permutation}이고 $\omega \in S^{\mathbb{N}}$이라면 아래와 같이 정의할 수 있다. 
\[
\mbox{$i$-th element of $\pi\omega$ $=$ $\omega$의 $\pi(i)$-th element}
\]
따라서 
\[
\big(\pi\omega\big)_i = \omega_{\pi(i)}
\]
와 같이 쓸 수 있다. 이 경우 array of realization:
\[
\omega_1,\omega_2,\dots
\]
이 $\pi$에 의해서 재정렬된다고 해석할 수 있다. (Note that $\omega_1=X_1(\omega)$)

\ck 따라서 array of realization 즉 어떠한 확률시행의 결과에 $\pi$를 취한다는 의미는 확률변수의 결과 중 유한개의 순서를 임의로 바꾼다는 것을 의미한다. 

\parablue{def} 어떠한 사건 $A$가 \emph{\textcolor{blue}{permutable}}하다는 것은 사건 $A$가 아래를 식을 만족한다는 의미이다. 
\[
A=\pi^{-1}(A)
\]
여기에서 $\pi^{-1}(A)=\{\omega:\pi\omega \in A\}$이다. 이 정의는 이해하기 그렇게 쉽지않다. 아래의 설명들을 참고하여 보자. 

\ck 우선 사건 $A$란 무엇인지 다시 생각해보자. 사건이란 확률시행의 결과를 재해석하여 구성한 어떠한 이벤트를 의미한다. 따라서 $A$는 예를들어 "주사위를 5번 던졌을때 짝수가 3번이상 나올 사건"과 같이 정의할 수 있다. 

\ck 이제 퍼뮤터블의 의미를 살펴보자. 어떠한 사건 $A$ 퍼뮤터블하다는 것은 확률변수의 결과 중 유한개의 순서를 임의로 바꾸어도 사건 $A$를 일관적으로 정의할 수 있다는 것을 의미한다. 

\ex 예를들어 보자. $X_1,X_2,\dots$이 $1$과 $-1$중 하나가 나오는 베르누이 시행이라고 하자. 만약 
\[
\mbox{$A$ : 랜덤변수들의 총합 즉 $X_1+X_2+X_3+\dots$ 가 음수일 사건 }
\]
이라 정의한다고 하자. 직관적으로 $A$는 확률변수들의 순서를 바꾸어서 상관없으므로 이럴 경우 사건 $A$를 퍼뮤터블 하다고 말한다. 
이 예제를 좀 더 수학적으로 표현하여 보자. 
\[
A=\{\omega : X_1(\omega)+X_2(\omega)+\dots<0\}
\]
$\pi$를 인덱스 1과 2를 바꾸는 변환이라고 하자. 그러면 
\[
\pi\omega=\pi(\omega_1,\omega_2,\dots)=(\omega_2,\omega_1,\dots)
\]
따라서 
\[
\omega \in A \Longleftrightarrow X_1(\omega)+X_2(\omega)+X_3(\omega)+\dots<0
\]
에 대응하는 것은 
\[
\pi\omega \in A \Longleftrightarrow X_2(\omega)+X_1(\omega)+X_3(\omega)+\dots<0
\]
결국 
\[
\pi^{-1}(A)=\{\omega : \pi \omega \in A\}=\{\omega : \omega \in A\}=A.
\]

\ex 편의상 $X_1,X_2,X_3$을 베르누이시행에서 관측하였다고 하자. 
\[
\mbox{$A$: $X_1+X_2+X_3=1$ 일 사건} 
\]
그러면 
\[
A=\{\omega: \bsX(\omega)=(1,0,0) \mbox{ or } \bsX(\omega)=(0,1,0) \mbox{ or } \bsX(\omega)=(0,0,1)\}
\]
이다. 단, $\bsX(\omega)=\big(X_1(\omega),X_2(\omega),X_3(\omega)\big)$. 이때 $A$는 퍼뮤터블하다. 아래와 같은 collection을 고려하여 보자. 
\[
{\cal E}=\{\emptyset,A,A^c,\Omega \}
\]
$\cal E$는 시그마필드이고 모든 원소가 퍼뮤터블하다. 

\parablue{def} ${\cal E}$를 \emph{\textcolor{blue}{exchangeable}} 시그마필드라고 말한다. 즉 어떠한 시그마필드를 구성하는 모든 사건이 퍼뮤터블하면 익스체인지어블하다고 말한다. 

\ex 
\ck 
퍼뮤터블이벤트에 대한 예제를 좀더 살펴보자. (교재에 있는 예제이다.) 아래의 사건은 퍼뮤터블하다. 
\[
\{\omega:S_n(\omega) \in B, ~i.o.\}
\]
위의 사건이 의미하는 것은 유한개의 $n$을 제외하고 모두 
\[
S_n \in B^c
\]
라는 의미이다. 왜냐하면 
\begin{align*}
& S_n \in B, ~i.o. \\
& \Longleftrightarrow S_n \in B^c, ~ a.b.f. \\ 
& \Longleftrightarrow S_n \in B^c, ~ \mbox{for all } n\geq N_0
\end{align*}
이기 때문이다. 

\ck 이제 편의상 확률변수열 $\{X_n\}$에서 유한개의 인덱스를 서로 바꾸어 확률변수열 $\{\tilde{X}_n\}$을 만들었다고 가정하자. 그리고 
\[
S_n=X_1+\dots+X_n=X_1(\omega)+\dots+X_n(\omega)
\]
이라고 하고 
\[
\tilde{S}_n=\tilde{X}_1+\dots+\tilde{X}_n=X_1(\pi\omega)+\dots+X_n(\pi\omega) 
\]
이라고 하자. 유한개의 $n$을 제외하면 
\[
S_n\equiv S_n(\omega)=S_n(\pi\omega)\equiv \tilde{S}_n
\]
임을 주장할 수 있다. 즉 
\[
\exists N_0 \quad s.t. \quad \forall n \geq N_0: ~~ S_n=\tilde{S}_n
\]
가 성립함을 보일 수 있다. 그도 그럴것이 $\pi$는 유한개의 인덱스를 서로 바꾸는 역할을 하므로 적당히 큰 수 $N_0$이상으로는 그 유한개의 바꿈이 없다고 보일수 있다. (이때 큰 수 $N_0$의 존재성은 "유한"개의 바꿈이라는 조건때문에 증명된다.) 예를들어 100을 1로 1을 100으로 바꾼다고 하자. 그러면 $N_0=100$이 존재하여 
\[
\forall n \geq 100: ~~ S_n=\tilde{S}_n
\]
라고 주장할 수 있다. 

\ck 이제 
\[
\exists N_0 \quad s.t. \quad \forall n \geq N_0: ~~ S_n=\tilde{S}_n
\]
를 이용하면 아래를 얻을 수 있다. (풀이라기보다 표현들을 익숙하게 하기 위해 한번 쓴 것)
\begin{align*}
& \{S_n(\omega) \in B, ~i.o.\} \\
& =\{S_n(\omega) \in B^c, ~a.b.f.\} \\ 
& =\{\tilde{S}_n(\omega) \in B^c, ~a.b.f.\} \\ 
& =\{S_n(\pi\omega) \in B^c, ~a.b.f.\} \\ 
& =\{S_n(\pi\omega) \in B, ~i.o.\}
\end{align*}
따라서 사건 $A=\{\omega:S_n(\omega) \in B, ~i.o.\}$에 대하여 
\[
\omega \in A \Longleftrightarrow \pi\omega \in A
\]
이다. 

\note $S_n(\omega)=S_n(\pi\omega)$은 그냥 모든 $n$에 대하여 성립하는것 아닌가? 하는 착각을 하지 않기를 바란다. 물론 
\[
S=X_1+X_2+X_3+\dots
\]
은 항상 
\[
S(\omega)=S(\pi\omega)
\]가 성립하지만 $S_n$은 그렇지 않다. 예를들어서 $\pi$를 1과 100의 인덱스를 서로 바꾸는 규칙이라고 하자. 
\begin{align*}
& S_1(\omega)=X_1 \mbox{ and } S_1(\pi\omega)=X_{100} \\& S_2(\omega)=X_1+X_2 \mbox{ and } S_2(\pi\omega)=X_{100}+X_2 \\
\dots
\end{align*}
이므로 $n<100$에 대하여서는 $S_n(\omega)\neq S_n(\pi\omega)$이다. 

\ex 아래의 사건도 퍼뮤터블하다. 
\[
\left\{\omega:\limsup_{n\to\infty}S_n(\omega)/c_n \leq 1\right\}
\]
이유는 적당히 큰 $n$에 대하여 $S_n(\omega)=S_n(\pi\omega)$라고 주장할 수 있기 때문이다.

\para{thm} 테일-시그마필드에 속하는 모든 이벤트는 퍼뮤터블하다. 이때 \textcolor{blue}{\emph{tail-$\sigma$-field}}는 아래와 같이 정의되는 시그마필드 ${\cal T}$ 이다. 
\[
{\cal T}=\bigcap_{n=1}^{\infty}{\cal F}'_n \quad \mbox{where ${\cal F}'_n=\sigma(X_n,X_{n+1},X_{n+2},\dots)$}
\]

\note 어떠한 이벤트 $A$가 $A\in{\cal T}$라는 것은 임의의 유한개의 realization의 결과를 \textbf{몰라도(=삭제하여도, 아무값이나 넣어도)} event $A$가 동일하게 정의될 수 있음을 의미한다. 

\ck 위의 정리의 역은 성립하지 않는다. 즉 퍼뮤터블한 사건이 항상 테일시그마 필드의 원소는 아니다. 아까 소개한 바 있는 아래의 사건
\[
\{\omega:S_n(\omega) \in B, ~i.o.\}
\]
은 ${\cal E}$에 속하지만(=퍼뮤터블하지만) ${\cal T}$에 속하지 않는(=테일시그마필드의 원소는아닌) 사건이다. 이를 이해하기 위해 좀 더 구체적인 예로 생각해보자. 아래와 같은 구조에서 확률변수가 생성된다고 하자. 
\begin{align*}
& X_1 \sim Ber(p) \\ 
& X_2 = -X_1  \\ 
& X_n = 0 \mbox{ for } n\geq 3.
\end{align*}
관측가능한 확률변수열 $\{X_n\}$은 아래의 2경우 뿐이다. 
\[
-1,1,0,0,0,0,\dots
\] 
\[
1,-1,0,0,0,0,\dots
\] 
사건 $A_n$를 아래와 같이 정의하자. 
\[
A_n=\{S_n(\omega) \in B, ~ i.o.\} \quad B=\{0\}
\] 
이라고 하자. 사건 $A_n$는 퍼뮤터블하다. 왜냐하면 확률변수열의 순서를 임의로 바꾸어도 적당히 큰 $n$에 대하여 (구체적으로는 $n\geq 2$에 대하여) 
\[
0=S_n=\tilde{S}_n=0
\]
이 성립하기 때문이다. 즉 $X_1+X_2+\dots+X_n=0$이라는 사실은 확률변수열의 순서를 아무리 바꾸어도 $n\geq 2$에서 항상 성립한다. 하지만 사건 $A_n$는 테일시그마필드의 원소가 아니다. 왜냐하면 $A_n$가 테일시그마필드의 원소라면 특정 값을 삭제해도 그 결과가 균일하게 정의되어야 하는데 첫번째 값 $X_1$을 삭제한다면 
\begin{align*}
& S_n=X_2+X_3+\dots+X_n= X_2 \\
& \Longrightarrow P(A_n)=P(S_n \in B,~i.o.)=P(X_2=0, ~i.o.)=0
\end{align*}
$X_1,X_2$를 모두 삭제하면 
\begin{align*}
& S_n=X_3+X_4+\dots+X_n=0 \\
& \Longrightarrow P(A_n)=P(S_n \in B,~i.o.)=P(S_n=0, ~i.o.)=1
\end{align*}
이 되어서 $A_n$을 균일하게 정의할 수 없다. 
\note 물론 사건 $A_n$이 테일시그마필드의 원소라면 특정 확률변수를 삭제하는 것이 아니라 아예 $\mathbb{R}$의 임의의 값으로 바꾸어친다쳐도 항상 $A_n$이 균일하게 정의 할 수 있어야 한다. 처음부터 이렇게 생각했으면 $A_n$이 테일시그마필드의 원소가 아님을 더 쉽게 보일 수 있다. 세번째 값 $X_3$을 100으로 바꿔친다면 
\begin{align*}
S_n=X_1+X_2+X_3+\dots+X_n=100.
\end{align*}
세번째 값 $X_3$을 200으로 바꾸면 
\begin{align*}
S_n=X_1+X_2+X_3+\dots+X_n=200.
\end{align*}
따라서 $A_n$을 균일하게 정의할 수 없다. 

\ck 참고로 위와 동일한 예제에서 
\[
A_n=\{S_n \in (-\infty, \infty) \}
\]
와 같이 정의하면 사건은 퍼뮤터블하고 테일시그마필드의 원소가 된다. 

\ck 만약에 확률변수열 $X_1,X_2,\dots$이 iid 라면 ${\cal E}$와 ${\cal T}$는 차이가 없다. 이게 바로 \textbf{휴이트-새비지}의 정리이다. 

\parared{Theorem 4.1.1. Hewitt-Savage 0-1 law.} 
(1) $X_1,X_2,\dots$ 이 i.i.d.이고 (2) $A \in {\cal E}$ 이라면 
\[
P(A) \in \{0,1\}
\]
이다. 

\parared{Theorem 4.1.2.} 
랜덤워크 $S_n$은 아래의 4개의 가능성밖에 없다. 
\para{i} $S_n=0$ for all $n$.
\para{ii} $S_n\to \infty$.
\para{iii} $S_n\to -\infty$.
\para{iv} $-\infty=\liminf S_n<\limsup S_n =\infty$.
이중에서 (iv)의 경우는 $S_n$이 $-\infty$와 $\infty$사이를 끝없이 진동하는 경우이다. 즉 $(-2)^{n}$와 같은 경우이다. 

\parablue{def}
${\cal F}_n=\sigma(X_1,\dots,X_n)$이라고 하자. 그리고 $N$을 $\{1,2,\dots,\}\cup\{\infty\}$중 하나의 값을 가지는 확률변수라고 하자. 만약에 아래가 성립한다면 $N$을 \emph{\textcolor{blue}{stopping time}} 이라고 부른다. 
\[
\mbox{for all $n<\infty$, $\{N=n\}\in {\cal F}_n$.}
\]



\subsubsection{6.1. Definitions}
\parablue{def} $(S,{\cal S})$를 measurable space라고 하자. $X_n:(\Omega,{\cal F}) \to (S,{\cal S})$이라고 하자. 편하게 
\[
(S,{\cal S})=(\mathbb{R},{\cal R})
\]
이라고 생각해도 무방하다. 어떠한 확률변수열 $\{X_n\}$이 filtration ${\cal F}_n:=\sigma(X_0,\dots,X_n)$에서 정의되어 있다고 하자. 확률변수열 $\{X_n\}$이 마코프체인이라는 것은 아래와 같이 정의한다. 
\begin{align*}
& \{X_n\} \mbox{ is Markovchain w.r.t. } {\cal F}_n \\
& \overset{def}{\Longleftrightarrow} \mbox{for all $B\in {\cal S}$:}\quad 
P(X_{n+1}\in B | {\cal F}_n)=P(X_{n+1}\in B | X_n)
\end{align*}

\note 확률변수열 $X_1,X_2,\dots$의 값이 바로 이전의 값에 의해서만 결정되면 마코프체인이라고 한다. 즉 $X_2$의 값을 알기 위해서는 $X_1$의 값에 대한 정보만 있으면 되고 $X_3$의 값을 알기 위해서는 $X_2$에 대한 정보만 있으면 될때 $X_1,X_2,\dots$을 마코프체인이라고 한다. 

\ck $4\times 4$ 그리드 세계를 가정하자. 
\[
\Omega=\{(1,1),\dots,(4,4)\}
\]
이고 
\[
S=\{1,2,3\dots,16\}
\]
이라고 하자. 확률변수 $X_1:(\Omega,{\cal F}) \to (S, {\cal S})$은 아래와 같이 정의할 수 있는 맵핑이라고 하자. 
\begin{align*}
& X_1\big((1,1)\big)=1\\ 
& X_2\big((1,2))=2\\ 
& \dots \\
& X_1\big((4,4)\big)=16
\end{align*}
따라서 
\[
X_1=1, X_2=2
\]
가 의미하는 것은 처음에는 $(1,1)$의 위치에 있다가 그다음에는 $(1,2)$의 위치로 이동하였다는 것을 의미한다. 이제 $(1,1)$의 위치에서 $(1,2)$의 위치로 이동하는 transition probability를 $p$라고 정의하자. 여기에서 $p$는 확률공간을 구성하는 $P$이지 pmf를 의미하는 것이 아님을 기억하자. $p$는 아래와 같이 정의할 수 있다. 
\[
p:(S,{\cal S}) \to \mathbb{R}
\]
기호로는 아래와 같이 쓴다. 
\[
p(x,A)
\]
여기에서 $x \in S$, $A \in {\cal S}$이다. 

\ex $x=1$, $A=\{1,2,5\}$이라고 하자. 
\begin{align*}
& x=1 \Leftrightarrow X(\omega)=1 \Leftrightarrow \omega=(1,1) \\ 
& A=\{1,2,5\} \Leftrightarrow \{\omega:X(\omega) \in A\}=\{(1,1),(1,2),(2,1)\}
\end{align*}
임을 주목하라. 따라서 
\[
p(x,A)
\] 
는 점 $(1,1)$에서 출발했는데 점 $(1,1),(1,2),(2,1)$중 하나에 도착할 확률이므로 
\[
p(x,A)=1
\]
이라고 볼 수 있다. 

\parablue{def} transition probability의 정의를 사용하면 아래를 만족하는 확률변수열 $\{X_n\}$을 마코프체인이라 정의할 수 있다. 
\[
P(X_{n+1}\in B|{\cal F}_n)=p_n(X_n,B)
\]
여기에서 $p_n$은 $n$번째에 어떠한 위치 $X_n$에서 $B$의 부분집합중 하나의 위치로 이동할 확률을 의미한다. 

\parared{결론1,2,3의 가정} 만약에 (1) $(S,{\cal S})$이 \textcolor{red}{\emph{nice space}} 이고 (2) $\{p_n\}$이 잘 정의되며 (3) $(S,{\cal S})$에서의 initional distribution $\mu$가 잘 정의된다고 하자. 

\parared{결론1} 일단 유한개의 확률변수열 $\{X_n\}$에 대하여 consistence set of finite dimensional distribution을 아래와 같이 잘 정의할 수 있다. 
\begin{align*}
& \mbox{Prob}(X_1\in B_1, X_2\in B_2, \dots, X_n \in B_n)\\
& = P(X_1\in B_1, X_2\in B_2, \dots, X_n \in B_n)\\
& = \int_{B_0}\mu(dx_0)\int_{B_1}p_0\big(x_0,\mu(dx_1)\big)\dots\int_{B_n}p_{n-1}\big(x_{n-1},\mu(dx_n)\big)
\end{align*}
\note 이때 $\mbox{Prob}(X_1\in B_1, X_2\in B_2, \dots, X_n \in B_n)$와 같은 표현은 이해하기 쉽지만 수학적으로 엄밀하지 않은 표현이다. 따라서 엄밀하게 하려면 아래의 공간에서 정의되는 확률측도 $P$를 사용하여 표현해야한다. 
\[
(S_0\times S_1 \dots \times S_n, {\cal S}_0\times {\cal S}_1 \dots \times {\cal S}_n)
\]
이 공간은 간단하게 아래와 같이 표현하기도 한다. 
\[
\left(S^{\{0,1,\dots,n\}},{\cal S}^{\{0,1,\dots,n\}}\right)
\]

\note 즉 결론1은 $\left(S^{\{0,1,\dots,n\}},{\cal S}^{\{0,1,\dots,n\}}\right)$에서의 확률측도 $P$는 (혹은 임의의 유한 확률변수열 $\{X_n\}$에 대한 확률측도 $P$는) 초기분포 $\mu$와 $p_n$만 잘 정의하면 모순없이 정의가능하다는 것을 의미한다.

\ck 하지만 무한일 경우에도 잘 정의될까? 

\parared{결론2} (1)-(3)의 가정하에 Kolmogorov's theorem은 확률변수열 $\{X_n\}$이 무한수열을 가지더라도 아래와 같은 확률이 잘 정의됨을 보여준다. 
\[
\mbox{Prob}(X_1\in B_1, X_2\in B_2, \dots, )
\]
즉 이는 $\mu$와 $\{p_n\}$만 잘 정의되면 위의 같은 확률들을 모순없이 정의할 수 있음을 의미한다. 교재에서는 유한인 경우와 구분하기 위해서 위의 확률을 표현하는 확률측도를 특별히 $P_{\mu}$라고 하였다. 즉 아래와 같이 써야 올바르다. 
\[
P_{\mu}(X_1\in B_1, X_2\in B_2, \dots)
\]
이때 $P_{\mu}$는 $\left(S^{\{0,1,\dots\}},{\cal S}^{\{0,1,\dots,\}}\right)$에서의 확률측도이다. 

\note 아래의 기호는 외우는 것이 좋겠다. 
\begin{align*}
& \mbox{Prob}(X_0\in B_0)= \int_{B_0} \mu(dx_0)\\ 
& \mbox{Prob}(X_0\in B_0,X_1 \in B_1)= \int_{B_0} \mu(d(x_0)) \int_{B_1} p\big(x_0,\mu(dx_1)\big) \\ 
\end{align*}

\ck 지금까지는 콜모고로프의 정리덕에 $P_{\mu}$가 잘 정의된다는 사실까지 살펴보았다. 즉 (1) 초기분포 $\mu$와 (2) transition 확률 $\{p_n\}$이 잘 정의되면 무한하게 눈을 쌓아도 $P_{\mu}$가 잘 정의된다. 

\parared{결론3, Thm 6.1.1} (1)-(3)의 조건하에 $\{X_n\}$이 마코프체인이 된다. 
\pf 아래의 기호를 정의하면서 증명을 시작하자. 
\parablue{notation}
$\mu=\delta_x$를 $x$에서의 포인트매스라고 하자. 그리고 기호 $P_x=P_{\delta_x}$라고 정의하자. $P_x$가 정의되면 아래와 같이 $P_{\mu}$를 정의할 수 있다. 
\[
P_{\mu}(A)=\int \mu(dx)P_x(A), \quad A \in {\cal S}^{\{0,1,\dots,\}}
\]

\parared{Thm 6.1.2, 결론1의 변형} (1)-(3)의 조건중 체크하기 까다로운 것은 (1)이다. 오히려 (1)의 조건대신에 $\{X_n\}$이 마코프체인임을 가정하면 결론2와 동일한 결과를 얻을 수 있다. 즉 (1) $\{X_n\}$이 마코프체인이고 (2) transition prob $\{p_n\}$이 주어졌고 (3) initional distribution $\mu$ 가 주어졌다면 \textcolor{red}{\emph{finite dimensional distribution}}이 아래와 같이 주어진다. 
\begin{align*}
& P(X_j \in B_j, 0\leq j \leq n) \\ 
& = P(X_1\in B_1, X_2\in B_2, \dots, X_n \in B_n)\\
& = \int_{B_0}\mu(dx_0)\int_{B_1}p_0\big(x_0,\mu(dx_1)\big)\dots\int_{B_n}p_{n-1}\big(x_{n-1},\mu(dx_n)\big)
\end{align*}

\subsubsection{6.2. Examples}

\subsubsection{6.3. Extensions of the Markov Property}

\subsubsection{6.4. Recurrence and Transience}

\subsubsection{6.5. Stationary Measures}
\ck 아래식을 만족하는 measure $\mu$를 stationary measure라고 한다. 
\[
\sum_x\mu(x)p(x,y)=\mu(y)
\]
\note $p(x,y)$: 노드 $x$에서 다음 노드 $y$로 이동할 확률
\note $\mu(x)$: 노드 $x$에 있을 확률
\note 따라서 stationary measure는 특정노드에 있을 확률을 측정하는 메져라 생각할 수 있다. 


\ck stationary measure(=stationary distribution)가 (1) 존재하고 (2) 유일하다는 것이 조사되었다고 하자. 이제 다음 관심사는 아래식을 만족하는 staionary distribution $\pi$이다. 
\[
\pi p =\pi
\]

\parablue{정리 6.5.6.} 

\parablue{정리 6.5.6.} $p$가 irreducible 하다는 것과 아래는 동치이다. 
\one .
\two stationary distribution이 존재한다. 
\three . 


\subsubsection{Asymptotic Behavior}
\parablue{레마 6.6.3.} $d_x=1$이라면 $m_0$보다 큰 모든 $m$에 대하여 
\[
p^m(x,x)>0
\]
를 만족시킬 수 있다. 

\parablue{정리 6.6.4.} $p$가 (1) irreducible 하고 (2) aperiodic 하며 (3) stationary distribution $\pi$를 가진다고 하자. 그러면 아래가 성립한다. 
\[
p^{n}(x,y) \to \pi(y) \quad as~ n\to \infty .
\]
\note $p$가 irreducible 인것만 보이면 stationary distribution $\pi$를 가진다는 것은 정리 6.5.6에 의해서 성립한다. 따라서 (1)-(2)만 조건으로 사용해도 위의 정리는 성립한다. 
\note $p$가 에이피리오딕하다는 의미는 모든 state가 $d_x=1$을 가진다는 것을 의미한다 .

\pf

\ck $S^2=S\times S$ 라고 하자. 
\ck 전이확률 $\bar{p}$를 $S\times S$에서 아래와 같이 정의하자. 
\[
\bar{p}\big((x_1,y_1),(x_2,y_2)\big)=p(x_1,x_2)p(y_1,y_2)
\]
\note 이는 각각의 coordinate가 독립적으로 움직인다는 것을 의미한다. 
\parared{step 1} 
\ck 먼저 $\bar{p}$가 이리듀시블임을 보이자. (이는 너무 당연해서 바보같은 증명으로 보이지만 정리의 에이피리오딕조건을 사용하는 유일한 과정이다.) 우선 $p$가 이리듀시블하다는 조건으로부터 아래를 만족하는 적당한 $K,L$이 존재함을 알 수 있다. 
\[
p^K(x_1,x_2)>0 \quad \mbox {and}\quad p^L(y_1,y_2)>0.
\]
그런데 레마 6.6.3에 의해서 $M$을 적당히 크게 설정한다면 아래를 만족시킬수 있음을 알 수 있다. 
\[
p^{L+M}(x_1,x_2)>0 \quad \mbox {and}\quad p^{K+M}(y_1,y_2)>0.
\]
따라서 아래가 성립한다. 
\[
\bar{p}^{K+L+M}\big((x_1,y_1),(x_2,y_2) \big)>0
\]

\parared{step 2} 
\ck 두 코디네이츠가 독립이므로 ${\bar p}$의 stationary distribution을 아래와 같이 정의할 수 있다. 
\[
\bar{\pi}(a,b)=\pi(a)\pi(b).
\]

\ck 정리 6.5.4에 의해서 $\bar{p}$의 stationary distribution이 존재한다는 것은 $\bar{p}$의 모든상태가 recurrent하다는 것을 의미한다. 

\ck $(X_n,Y_n)$을 $S \times S$에서의 체인이라고 하자. 

\ck $T$를 이 체인이 처음으로 대각 $\{(y,y)\in S\}$을 치는 시간이라고 하자. 

\ck $T(x,x)$를 $(x,x)$를 hit하는 시간이라고하자. 

\ck $\bar{p}$가 (1) irreducible 하고 (2) recurrent 하므로 $T(x,x)<\infty ~ a.s.$ 이고 따라서 $T <\infty ~ a.s.$ 이다. 

\parared{step 3} 

\ck 우선 두개의 코디네이트 $(X_n,Y_n)$가 $\{T\leq n\}$에서 같은 분포를 가진다는 것을 관찰하자. 

\ck $(X_n,Y_n)$이 첫 교차점을 가지는 시간과 장소를 고려하여보자. 마코프성질을 이용하면 
\[
P(X_n=y,T\leq n)= \sum_{m=1}^{n}\sum_xP(T=m,X_m=x,X_n=y)
\]


\subsubsection{6.6. Periodicity, Tail $\sigma$-field}

\subsubsection{6.7. General State}


\end{document}

