\documentclass[12pt,oneside,english]{book}
\usepackage{babel}
\usepackage[utf8]{inputenc}
\usepackage[T1]{fontenc}
\usepackage{color}
\definecolor{marron}{RGB}{60,30,10}
\definecolor{darkblue}{RGB}{0,0,80}
\definecolor{lightblue}{RGB}{80,80,80}
\definecolor{darkgreen}{RGB}{0,80,0}
\definecolor{darkgray}{RGB}{0,80,0}
\definecolor{darkred}{RGB}{80,0,0}
\definecolor{shadecolor}{rgb}{0.97,0.97,0.97}
\usepackage{graphicx}
\usepackage{wallpaper}
\usepackage{wrapfig,booktabs}

\usepackage{fancyhdr}
\usepackage{lettrine}
\input Acorn.fd

\renewcommand{\familydefault}{pplj} 
\usepackage[
final,
stretch=10,
protrusion=true,
tracking=true,
spacing=on,
kerning=on,
expansion=true]{microtype}

%\setlength{\parskip}{1.5ex plus 0.2ex minus 0.2ex}


\usepackage{fourier-orns}
\newcommand{\dash}{\vspace{2em}\noindent \textcolor{darkgray}{\hrulefill~ \raisebox{-2.5pt}[10pt][10pt]{\leafright \decofourleft \decothreeleft  \aldineright \decotwo \floweroneleft \decoone   \floweroneright \decotwo \aldineleft\decothreeright \decofourright \leafleft} ~  \hrulefill \\ \vspace{2em}}}
\newcommand{\rdash}{\noindent \textcolor{darkgray}{ \raisebox{-1.9pt}[10pt][10pt]{\leafright} \hrulefill \raisebox{-1.9pt}[10pt][10pt]{\leafright \decofourleft \decothreeleft  \aldineright \decotwo \floweroneleft \decoone}}}
\newcommand{\ldash}{\textcolor{darkgray}{\raisebox{-1.9pt}[10pt][10pt]{\decoone \floweroneright \decotwo \aldineleft \decothreeright \decofourright \leafleft} \hrulefill \raisebox{-1.9pt}[10pt][10pt]{\leafleft}}}

\fancyhf{}

\renewcommand{\chaptermark}[1]{\markboth{#1}{}}
\renewcommand{\sectionmark}[1]{\markright{#1}}

\newcommand{\estcab}[1]{\itshape\textcolor{marron}{\nouppercase #1}}

\fancyhead[LO]{\estcab{\rightmark}} 
\fancyhead[RO]{\estcab{\leftmark}}
\fancyhead[RO]{\bf\nouppercase{ \leftmark}}
\fancyfoot[RO]{ \leafNE  ~~ \bf \thepage}

\newenvironment{Section}[1]
{\section{\vspace{0ex}#1}}
{\vspace{12pt}\centering ------- \decofourleft\decofourright ------- \par}

\usepackage{lipsum}
\setlength{\parindent}{1em} % Sangría española
\pagestyle{fancy}

\renewcommand{\footnoterule}{\noindent\textcolor{marron}{\decosix \raisebox{2.9pt}{\line(1,0){100}} \lefthand} \vspace{.5em} }
\usepackage[hang,splitrule]{footmisc}
\addtolength{\footskip}{0.5cm}
\setlength{\footnotemargin}{0.3cm}
\setlength{\footnotesep}{0.4cm} 

\usepackage{chngcntr}
\counterwithout{figure}{chapter}
\counterwithout{table}{chapter}

\usepackage{kotex}
\usepackage{amsthm} 
\usepackage{amsmath} 
\usepackage{amsfonts}
\usepackage{enumerate} 
\usepackage{cite}
\usepackage{graphics} 
\usepackage{graphicx,lscape} 
\usepackage{subcaption}
\usepackage{algpseudocode}
\usepackage{algorithm}
\usepackage{titlesec}
\usepackage{cite, url}
\usepackage{amssymb}

\def\ck{\paragraph{\Large$\bullet$}\Large}
\def\goal{\paragraph{\Large(목표)}\Large}
\def\observe{\paragraph{\Large(관찰)}\Large}
\def\assume{\paragraph{\Large(가정)}\Large}
\def\summary{\paragraph{\Large(요약)}\Large}
\def\EX{\paragraph{\Large(예제)}\Large}
\def\guess{\paragraph{\Large(추측)}\Large}
\def\thus{\paragraph{\Large(결론)}\Large}

\def\prob{\paragraph{\Large(문제)}\Large}
\def\sol{\paragraph{\Large(sol)}\Large}
\def\pf{\paragraph{\Large(pf)}\Large}

\def\dfn{\paragraph{\Large(정의)}\Large}
\def\thm{\paragraph{\Large(정리)}\Large}
\def\lem{\paragraph{\Large(레마)}\Large}
\def\promise{\paragraph{\Large(약속)}\Large}
\def\property{\paragraph{\Large(특징)}\Large}
\def\fl{\paragraph{\Large(느낌)}\Large}
\def\memo{\paragraph{\Large(암기)}\Large}

\def\note{\paragraph{\Large\textit{\underline{note:}}}\Large}
\def\ex{\paragraph{\Large\textit{example:}}\Large}


\def\one{\paragraph{\Large(1)}\Large}
\def\two{\paragraph{\Large(2)}\Large}
\def\three{\paragraph{\Large(3)}\Large}
\def\four{\paragraph{\Large(4)}\Large}
\def\five{\paragraph{\Large(5)}\Large}
\def\six{\paragraph{\Large(6)}\Large}
\def\seven{\paragraph{\Large(7)}\Large}
\def\eight{\paragraph{\Large(8)}\Large}
\def\nine{\paragraph{\Large(9)}\Large}
\def\ten{\paragraph{\Large(10)}\Large}

\def\cka{\paragraph{\Large(a)}\Large}
\def\ckb{\paragraph{\Large(b)}\Large}
\def\ckc{\paragraph{\Large(c)}\Large}
\def\ckd{\paragraph{\Large(d)}\Large}
\def\cke{\paragraph{\Large(e)}\Large}
\def\ckf{\paragraph{\Large(f)}\Large}
\def\ckg{\paragraph{\Large(g)}\Large}
\def\ckh{\paragraph{\Large(h)}\Large}
\def\cki{\paragraph{\Large(i)}\Large}
\def\ckj{\paragraph{\Large(j)}\Large}

\newcommand{\bld}[1]{\mbox{\boldmath $#1$}}
\newcommand{\bs}[1]{\mbox{\boldmath $#1$}}

\newcommand{\bsa}{\mbox{\boldmath $a$}}
\newcommand{\bsb}{\mbox{\boldmath $b$}}
\newcommand{\bsc}{\mbox{\boldmath $c$}}
\newcommand{\bsd}{\mbox{\boldmath $d$}}
\newcommand{\bse}{\mbox{\boldmath $e$}}
\newcommand{\bsf}{\mbox{\boldmath $f$}}
\newcommand{\bsg}{\mbox{\boldmath $g$}}
\newcommand{\bsh}{\mbox{\boldmath $h$}}
\newcommand{\bsi}{\mbox{\boldmath $i$}}
\newcommand{\bsj}{\mbox{\boldmath $j$}}
\newcommand{\bsk}{\mbox{\boldmath $k$}}
\newcommand{\bsl}{\mbox{\boldmath $l$}}
\newcommand{\bsm}{\mbox{\boldmath $m$}}
\newcommand{\bsn}{\mbox{\boldmath $n$}}
\newcommand{\bso}{\mbox{\boldmath $o$}}
\newcommand{\bsp}{\mbox{\boldmath $p$}}
\newcommand{\bsq}{\mbox{\boldmath $q$}}
\newcommand{\bsr}{\mbox{\boldmath $r$}}
\newcommand{\bss}{\mbox{\boldmath $s$}}
\newcommand{\bst}{\mbox{\boldmath $t$}}
\newcommand{\bsu}{\mbox{\boldmath $u$}}
\newcommand{\bsv}{\mbox{\boldmath $v$}}
\newcommand{\bsw}{\mbox{\boldmath $w$}}
\newcommand{\bsx}{\mbox{\boldmath $x$}}
\newcommand{\bsy}{\mbox{\boldmath $y$}}
\newcommand{\bsz}{\mbox{\boldmath $z$}}

\newcommand{\bfa}{\mbox{$\bf{a}$}}
\newcommand{\bfb}{\mbox{$\bf{b}$}}
\newcommand{\bfc}{\mbox{$\bf{c}$}}
\newcommand{\bfd}{\mbox{$\bf{d}$}}
\newcommand{\bfe}{\mbox{$\bf{e}$}}
\newcommand{\bff}{\mbox{$\bf{f}$}}
\newcommand{\bfg}{\mbox{$\bf{g}$}}
\newcommand{\bfh}{\mbox{$\bf{h}$}}
\newcommand{\bfi}{\mbox{$\bf{i}$}}
\newcommand{\bfj}{\mbox{$\bf{j}$}}
\newcommand{\bfk}{\mbox{$\bf{k}$}}
\newcommand{\bfl}{\mbox{$\bf{l}$}}
\newcommand{\bfm}{\mbox{$\bf{m}$}}
\newcommand{\bfn}{\mbox{$\bf{n}$}}
\newcommand{\bfo}{\mbox{$\bf{o}$}}
\newcommand{\bfp}{\mbox{$\bf{p}$}}
\newcommand{\bfq}{\mbox{$\bf{q}$}}
\newcommand{\bfr}{\mbox{$\bf{r}$}}
\newcommand{\bfs}{\mbox{$\bf{s}$}}
\newcommand{\bft}{\mbox{$\bf{t}$}}
\newcommand{\bfu}{\mbox{$\bf{u}$}}
\newcommand{\bfv}{\mbox{$\bf{v}$}}
\newcommand{\bfw}{\mbox{$\bf{w}$}}
\newcommand{\bfx}{\mbox{$\bf{x}$}}
\newcommand{\bfy}{\mbox{$\bf{y}$}}
\newcommand{\bfz}{\mbox{$\bf{z}$}}

\newcommand{\bsA}{\mbox{$\boldmath{A}$}}
\newcommand{\bsB}{\mbox{$\boldmath{B}$}}
\newcommand{\bsC}{\mbox{$\boldmath{C}$}}
\newcommand{\bsD}{\mbox{$\boldmath{D}$}}
\newcommand{\bsE}{\mbox{$\boldmath{E}$}}
\newcommand{\bsF}{\mbox{$\boldmath{F}$}}
\newcommand{\bsG}{\mbox{$\boldmath{G}$}}
\newcommand{\bsH}{\mbox{$\boldmath{H}$}}
\newcommand{\bsI}{\mbox{$\boldmath{I}$}}
\newcommand{\bsJ}{\mbox{$\boldmath{J}$}}
\newcommand{\bsK}{\mbox{$\boldmath{K}$}}
\newcommand{\bsL}{\mbox{$\boldmath{L}$}}
\newcommand{\bsM}{\mbox{$\boldmath{M}$}}
\newcommand{\bsN}{\mbox{$\boldmath{N}$}}
\newcommand{\bsO}{\mbox{$\boldmath{O}$}}
\newcommand{\bsP}{\mbox{$\boldmath{P}$}}
\newcommand{\bsQ}{\mbox{$\boldmath{Q}$}}
\newcommand{\bsR}{\mbox{$\boldmath{R}$}}
\newcommand{\bsS}{\mbox{$\boldmath{S}$}}
\newcommand{\bsT}{\mbox{$\boldmath{T}$}}
\newcommand{\bsU}{\mbox{$\boldmath{U}$}}
\newcommand{\bsV}{\mbox{$\boldmath{V}$}}
\newcommand{\bsW}{\mbox{$\boldmath{W}$}}
\newcommand{\bsX}{\mbox{$\boldmath{X}$}}
\newcommand{\bsY}{\mbox{$\boldmath{Y}$}}
\newcommand{\bsZ}{\mbox{$\boldmath{Z}$}}

\newcommand{\bfA}{\mbox{$\bf{A}$}}
\newcommand{\bfB}{\mbox{$\bf{B}$}}
\newcommand{\bfC}{\mbox{$\bf{C}$}}
\newcommand{\bfD}{\mbox{$\bf{D}$}}
\newcommand{\bfE}{\mbox{$\bf{E}$}}
\newcommand{\bfF}{\mbox{$\bf{F}$}}
\newcommand{\bfG}{\mbox{$\bf{G}$}}
\newcommand{\bfH}{\mbox{$\bf{H}$}}
\newcommand{\bfI}{\mbox{$\bf{I}$}}
\newcommand{\bfJ}{\mbox{$\bf{J}$}}
\newcommand{\bfK}{\mbox{$\bf{K}$}}
\newcommand{\bfL}{\mbox{$\bf{L}$}}
\newcommand{\bfM}{\mbox{$\bf{M}$}}
\newcommand{\bfN}{\mbox{$\bf{N}$}}
\newcommand{\bfO}{\mbox{$\bf{O}$}}
\newcommand{\bfP}{\mbox{$\bf{P}$}}
\newcommand{\bfQ}{\mbox{$\bf{Q}$}}
\newcommand{\bfR}{\mbox{$\bf{R}$}}
\newcommand{\bfS}{\mbox{$\bf{S}$}}
\newcommand{\bfT}{\mbox{$\bf{T}$}}
\newcommand{\bfU}{\mbox{$\bf{U}$}}
\newcommand{\bfV}{\mbox{$\bf{V}$}}
\newcommand{\bfW}{\mbox{$\bf{W}$}}
\newcommand{\bfX}{\mbox{$\bf{X}$}}
\newcommand{\bfY}{\mbox{$\bf{Y}$}}
\newcommand{\bfZ}{\mbox{$\bf{Z}$}}

\DeclareMathOperator*{\argmin}{argmin} %\usepackage{amsmath}를 써야지 정의가능함. 
\DeclareMathOperator*{\argmax}{argmax} %\usepackage{amsmath}를 써야지 정의가능함. 

\usepackage{titlesec}
\titleformat*{\section}{\huge\bfseries}
\titleformat*{\subsection}{\huge\bfseries}
\titleformat*{\subsubsection}{\huge\bfseries}
\titleformat*{\paragraph}{\huge\bfseries}
\titleformat*{\subparagraph}{\huge\bfseries}


\titleclass{\part}{top}
\titleformat{\part}[display]
  {\normalfont\huge\bfseries}{\centering\partname\ \thepart}{20pt}{\Huge\centering}
\titlespacing*{\part}{0pt}{50pt}{40pt}
\titleclass{\chapter}{straight}
\titleformat{\chapter}[display]
  {\normalfont\huge\bfseries}{\chaptertitlename\ \thechapter}{20pt}{\Huge}
\titlespacing*{\chapter} {0pt}{50pt}{40pt}

\newcommand*\initfamily{\usefont{U}{Acorn}{xl}{n}}
\usepackage[left=10px,right=10px,top=10px,bottom=10px,paperwidth=8in,paperheight=20in]{geometry}

\usepackage{geometry}
\geometry{
tmargin=3cm, 
bmargin=3cm, 
lmargin=1cm, 
rmargin=1cm,
headheight=1.5cm,
headsep=0.8cm,
footskip=0.5cm}
\linespread{1.7}
\begin{document}
\subsubsection{Big-$O_p$, little-$o_p$ 와 Delta-method 활용예제.}
\ck 이번에는 Big-$O_p$, little-$o_p$ 와 Delta-method를 활용한 예제를 살펴보겠다. 
\ck 김우철 수리통계학 교재의 연습문제를 참고함. 

\dash

\subsubsection{연습문제 5.14.}
\ck $\begin{pmatrix}X_1 \\ Y_1 \end{pmatrix},\dots,\begin{pmatrix}X_n \\ Y_n \end{pmatrix} \overset{idd}{\sim} F$ with
\one $EX_1=EY_1=0$,
\two $EX_1^2=EY_1^2=1$,
\three $EX_1Y_1=\rho$,
\four $EX_1^4<\infty$, $EY_1^4<\infty$. 

\ck $\hat{\rho}_n=\frac{\sum_{i=1}^{n}(X_i-\bar{X})(Y_i-\bar{Y})}{\sqrt{\sum_{i=1}^{n}(X_i-\bar{X})^2}\sqrt{\sum_{i=1}^{n}(Y_i-\bar{Y})^2}}.$ 

\ck 아래와 같은 기호를 약속하자. 
\one $\bar{X}=\frac{1}{n}\sum_{i=1}^{n}X_i$.
\two $\bar{X^2}=\frac{1}{n}\sum_{i=1}^{n}X_i^2$.
\three $\overline{XY}=\frac{1}{n}\sum_{i=1}^{n}X_iY_i$.

\rdash 

\cka $\sqrt{n} \bar{X}\bar{Y} \overset{p}{\to} 0$ 임을 보여라.  

\sol 

\ck $\sqrt{n}\bar{X}=O_p(1)$, $\bar{Y}=o_p(1)$ 이므로 자명함. 여기에서 $\sqrt{n}\bar{X}=O_p(1)$ 는 CLT에 의해서 성립하고 $\bar{Y}=o_p(1)$ 는 큰수의 법칙 즉 WLLN에 의해서 성립한다. 

\note 추가적으로 $\bar{XY}-\bar{X}\bar{Y}=\bar{XY}+o_p(\frac{1}{\sqrt{n}})$ 임도 알 수 있다. 

\rdash 

\ckb 아래가 성립함을 보여라. 
\[
\sqrt{m_2-m_1^2} - \sqrt{m_2}=\frac{-m_1^2}{\sqrt{m_2-m_1^2}+\sqrt{m_2}}
\]
이때 $m_1=\bar{X}$, $m_2=\bar{X^2}$ 라고 하자. 

\sol (자명하다)

\rdash 

\note 추가적으로 아래를 확인가능하다. 
\[
\sqrt{m_2-m_1^2} =\sqrt{m_2}+\frac{-m_1^2}{\sqrt{m_2-m_1^2}+\sqrt{m_2}}=\sqrt{m_2}+o_p\left(\frac{1}{\sqrt{n}}\right)
\]
\pf 
\ck $\sqrt{n}m_1^2=\sqrt{n}m_1 m_1 = O_p(1)o_p(1)=o_p(1)$ $\quad \therefore m_1^2=o_p(\frac{1}{\sqrt{n}})$.

\ck $m_2=1+o_p(1)$ 이다. 즉 $m_2 \overset{p}{\to} 1$이다. 또한 $m_1^2=o_p(\frac{1}{\sqrt{n}})$ 이다. 따라서 
\[
\frac{-m_1^2}{\sqrt{m_2-m_1^2}+\sqrt{m_2}}=o_p\left(\frac{1}{\sqrt{n}}\right)
\]
이다. 왜냐하면 분모는 1로 확률수렴하고 (continuous mappting thm) 분자는 $o_p(\frac{1}{\sqrt{n}})$이기 때문. 

\rdash 

\ckc 아래가 성립함을 보여라. 
\[
\sqrt{m_2}-1=\frac{m_2-1}{2}+o_p\left(\frac{1}{\sqrt{n}}\right)
\]
\sol 

\ck 아래식을 관찰하라. 
\[
\sqrt{m_2}-1=\frac{m_2-1}{\sqrt{m_2}+1}=\frac{m_2-1}{2}+\left(\frac{1}{\sqrt{m_2}+1}-\frac{1}{2}\right)(m_2-1)
\]

\ck 이제 아래를 보이면 된다. 
\[
\left(\frac{1}{\sqrt{m_2}+1}-\frac{1}{2}\right)(m_2-1):=r_{3n}=o_p\left(\frac{1}{\sqrt{n}}\right)
\]
따라서 아래를 보이면 된다. 
\[
\sqrt{n}\left(\frac{1}{\sqrt{m_2}+1}-\frac{1}{2}\right)(m_2-1)=o_p(1)
\]

\ck 그런데 CLT에 의해서 $\sqrt{n}(m_2-1)=O_p(1)$ 가 성립한다. 

\ck continuous mapping thm에 의해서 아래가 성립한다. 
\[
m_2 \overset{p}{\to} 1 \quad \Longrightarrow 
\quad \left(\frac{1}{\sqrt{m_2}+1}-\frac{1}{2}\right) \overset{p}{\to} 0
\]

\ck 따라서 
\[
\left(\frac{1}{\sqrt{m_2}+1}-\frac{1}{2}\right)=o_p(1)
\]

\ck 따라서 증명이 끝난다. 

\rdash 

\ckd 아래를 보여라. 
\[
c_1-m_1s_1-\rho\sqrt{m_2-m_1^2}\sqrt{s_2-s_1^2}= c_1-\rho\frac{m_2+s_2}{2}+o_p\left(\frac{1}{\sqrt{n}}\right)
\]
이때 $c_1=\bar{XY}$, $s_1=\bar{Y}$, $s_2=\bar{Y^2}$이다. 

\sol 

\ck (b)의 결과로부터 아래가 성립한다. 
\[
\sqrt{m_2-m_1^2}=\sqrt{m_2}+o_p\left(\frac{1}{\sqrt{n}}\right)
\]

\ck (c)의 결과로부터 아래가 성립한다. 
\[
\sqrt{m_2}-1=\frac{m_2-1}{2}+o_p\left(\frac{1}{\sqrt{n}}\right)
\]

\ck 둘을 종합하면 
\[
\sqrt{m_2-m_1^2}=1+\frac{m_2-1}{2}+o_p\left(\frac{1}{\sqrt{n}}\right)
\]

\ck 따라서 
\[
\sqrt{m_2-m_1^2}\sqrt{s_2-s_1^2}
\]
는 아래식들의 합이다. 
\one $1+\frac{s_2-1}{2}+o_p(\frac{1}{\sqrt{n}})$
\two $\frac{m_2-1}{2}\left(1+\frac{s_2-1}{2}+o_p(\frac{1}{\sqrt{n}})\right)=\frac{m_2-1}{2}+O_p(\frac{1}{\sqrt{n}})O_p(\frac{1}{\sqrt{n}})+o_p(\frac{1}{\sqrt{n}})$
\note $O_p(\frac{1}{\sqrt{n}})O_p(\frac{1}{\sqrt{n}})=\frac{1}{\sqrt{n}}\frac{1}{\sqrt{n}}O_p(1)=\frac{1}{\sqrt{n}}o_p(1)=o_p(\frac{1}{\sqrt{n}})$
\two 따라서 결국 $\frac{m_2-1}{2}\left(1+\frac{s_2-1}{2}+o_p(\frac{1}{\sqrt{n}})\right)=\frac{m_2-1}{2}+o_p(\frac{1}{\sqrt{n}}).$
\three $o_p(1/\sqrt{n})\left(1+\frac{s_2-1}{2}+o_p(1/\sqrt{n})\right)=o_p(\frac{1}{\sqrt{n}})$

\ck 따라서 
\[
\sqrt{m_2-m_1^2}\sqrt{s_2-s_1^2}=1+\frac{s_2-1}{2}+\frac{m_2-1}{2}+o_p\left(\frac{1}{\sqrt{n}}\right)
\]
정리하면 
\[
\sqrt{m_2-m_1^2}\sqrt{s_2-s_1^2}=\frac{s_2+m_2}{2}+o_p\left(\frac{1}{\sqrt{n}}\right)
\]
\ck 한편 (a) 로부터 
\[
m_1s_1=\bar{X}\bar{Y}=o_p\left(\frac{1}{\sqrt{n}}\right)
\]

\ck 따라서 증명이 끝난다. 

\rdash 

\cke $\sqrt{n}(\hat{\rho}_n-\rho)=\sqrt{n}\bar{W}+o_p(1)$ 임을 보여라. 단 
\[
W=XY-\frac{\rho(X^2+Y^2)}{2}.
\]

\sol 

\ck ${\bar W}=\bar{XY}-\frac{\rho({\bar X}^2+{\bar Y}^2)}{2}=c_1-\frac{\rho(m_2+s_2)}{2}$.

\ck $\sqrt{n}(\hat{\rho}_n-\rho)=\sqrt{n}\times \frac{c_1-m_1s_1-\rho\sqrt{m_2-m_1^2}\sqrt{s_2-s_1^2}}{\sqrt{m_2-m_1^2}\sqrt{s_2-s_1^2}}$ 

\ck (d)에 의해서 $\mbox{"분자"}=c_1-\frac{\rho(m_2+s_2)}{2}+o_p(\frac{1}{\sqrt{n}})$

\ck 분모는 1로 확률수렴한다. 

\ck 따라서 
\[
\sqrt{n}(\hat{\rho}_n-\rho)=\sqrt{n}\times \frac{c_1-\frac{\rho(m_2+s_2)}{2}+o_p(\frac{1}{\sqrt{n}})}{1+o_p(1)}= \frac{\sqrt{n}\bar{W}+o_p(1)}{1+o_p(1)}=\sqrt{n}\bar{W}+o_p(1)
\]


\end{document}
