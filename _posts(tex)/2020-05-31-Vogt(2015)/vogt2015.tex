\documentclass[12pt,oneside,english]{book}
\usepackage{babel}
\usepackage[utf8]{inputenc}
\usepackage[T1]{fontenc}
\usepackage{color}
\definecolor{marron}{RGB}{60,30,10}
\definecolor{darkblue}{RGB}{0,0,80}
\definecolor{lightblue}{RGB}{80,80,80}
\definecolor{darkgreen}{RGB}{0,80,0}
\definecolor{darkgray}{RGB}{0,80,0}
\definecolor{darkred}{RGB}{80,0,0}
\definecolor{shadecolor}{rgb}{0.97,0.97,0.97}
\usepackage[demo]{graphicx}
\usepackage{wallpaper}
\usepackage{wrapfig,booktabs}

\usepackage{fancyhdr}
\usepackage{lettrine}
\input Acorn.fd
\newcommand*\initfamily{\usefont{U}{Acorn}{xl}{n}}
 \usepackage[left=10px,right=10px,top=10px,bottom=10px,paperwidth=8in,paperheight=165in]{geometry}

\usepackage{geometry}
\geometry{
tmargin=5cm, 
bmargin=3cm, 
lmargin=1cm, 
rmargin=1cm,
headheight=1.5cm,
headsep=0.8cm,
footskip=0.5cm}

\renewcommand{\familydefault}{pplj} 
\usepackage[
final,
stretch=10,
protrusion=true,
tracking=true,
spacing=on,
kerning=on,
expansion=true]{microtype}

%\setlength{\parskip}{1.5ex plus 0.2ex minus 0.2ex}


\usepackage{fourier-orns}
\newcommand{\dash}{\vspace{2em}\noindent \textcolor{darkgray}{\hrulefill~ \raisebox{-2.5pt}[10pt][10pt]{\leafright \decofourleft \decothreeleft  \aldineright \decotwo \floweroneleft \decoone   \floweroneright \decotwo \aldineleft\decothreeright \decofourright \leafleft} ~  \hrulefill \\ \vspace{2em}}}
\newcommand{\rdash}{\noindent \textcolor{darkgray}{ \raisebox{-1.9pt}[10pt][10pt]{\leafright} \hrulefill \raisebox{-1.9pt}[10pt][10pt]{\leafright \decofourleft \decothreeleft  \aldineright \decotwo \floweroneleft \decoone}}}
% \newcommand{\ldash}{\textcolor{darkgray}{\raisebox{-1.9pt}[10pt][10pt]{\decoone \floweroneright \decotwo \aldineleft \decothreeright \decofourright \leafleft} \hrulefill \raisebox{-1.9pt}[10pt][10pt]{\leafleft}}}


\fancyhf{}

\renewcommand{\chaptermark}[1]{\markboth{#1}{}}
\renewcommand{\sectionmark}[1]{\markright{#1}}

\newcommand{\estcab}[1]{\itshape\textcolor{marron}{\nouppercase #1}}

\fancyhead[LO]{\estcab{\rightmark}} 
\fancyhead[RO]{\estcab{\leftmark}}
\fancyhead[RO]{\bf\nouppercase{ \leftmark}}
\fancyfoot[RO]{ \leafNE  ~~ \bf \thepage}

\newenvironment{Section}[1]
{\section{\vspace{0ex}#1}}
{\vspace{12pt}\centering ------- \decofourleft\decofourright ------- \par}

\usepackage{lipsum}
\setlength{\parindent}{1em} % Sangría española
\pagestyle{fancy}

\renewcommand{\footnoterule}{\noindent\textcolor{marron}{\decosix \raisebox{2.9pt}{\line(1,0){100}} \lefthand} \vspace{.5em} }
\usepackage[hang,splitrule]{footmisc}
\addtolength{\footskip}{0.5cm}
\setlength{\footnotemargin}{0.3cm}
\setlength{\footnotesep}{0.4cm} 

\usepackage{chngcntr}
\counterwithout{figure}{chapter}
\counterwithout{table}{chapter}

\usepackage{kotex}
\usepackage{amsthm} 
\usepackage{amsmath} 
\usepackage{amsfonts}
\usepackage{enumerate} 
\usepackage{cite}
\usepackage{graphics} 
\usepackage{graphicx,lscape} 
\usepackage{subcaption}
\usepackage{algpseudocode}
\usepackage{algorithm}
\usepackage{titlesec}
\usepackage{cite, url}
\usepackage{amssymb}

\def\ck{\paragraph{\Large$\bullet$}\Large}
\def\goal{\paragraph{\Large(목표)}\Large}
\def\observe{\paragraph{\Large(관찰)}\Large}
\def\assume{\paragraph{\Large(가정)}\Large}
\def\summary{\paragraph{\Large(요약)}\Large}
\def\EX{\paragraph{\Large(예제)}\Large}
\def\guess{\paragraph{\Large(추측)}\Large}
\def\thus{\paragraph{\Large(결론)}\Large}
\def\prob{\paragraph{\Large(문제)}\Large}
\def\sol{\paragraph{\Large(해결)}\Large}
\def\dfn{\paragraph{\Large(정의)}\Large}
\def\thm{\paragraph{\Large(정리)}\Large}
\def\lem{\paragraph{\Large(레마)}\Large}
\def\promise{\paragraph{\Large(약속)}\Large}
\def\property{\paragraph{\Large(특징)}\Large}
\def\note{\paragraph{\Large\textit{\underline{note:}}}\Large}
\def\ex{\paragraph{\Large\textit{example:}}\Large}

\def\one{\paragraph{\Large(1)}\Large}
\def\two{\paragraph{\Large(2)}\Large}
\def\three{\paragraph{\Large(3)}\Large}
\def\four{\paragraph{\Large(4)}\Large}
\def\five{\paragraph{\Large(5)}\Large}
\def\six{\paragraph{\Large(6)}\Large}
\def\seven{\paragraph{\Large(7)}\Large}
\def\eight{\paragraph{\Large(8)}\Large}
\def\nine{\paragraph{\Large(9)}\Large}
\def\ten{\paragraph{\Large(10)}\Large}

\newcommand{\bld}[1]{\mbox{\boldmath $#1$}}

\DeclareMathOperator*{\argmin}{arg\,min} %\usepackage{amsmath}를 써야지 정의가능함. 
\DeclareMathOperator*{\argmax}{arg\,max} %\usepackage{amsmath}를 써야지 정의가능함. 

\usepackage{titlesec}
\titleformat*{\section}{\huge\bfseries}
\titleformat*{\subsection}{\huge\bfseries}
\titleformat*{\subsubsection}{\huge\bfseries}
\titleformat*{\paragraph}{\huge\bfseries}
\titleformat*{\subparagraph}{\huge\bfseries}

\titleclass{\part}{top}
\titleformat{\part}[display]
  {\normalfont\huge\bfseries}{\centering\partname\ \thepart}{20pt}{\Huge\centering}
\titlespacing*{\part}{0pt}{50pt}{40pt}
\titleclass{\chapter}{straight}
\titleformat{\chapter}[display]
  {\normalfont\huge\bfseries}{\chaptertitlename\ \thechapter}{20pt}{\Huge}
\titlespacing*{\chapter} {0pt}{50pt}{40pt}

\begin{document}
% \TileWallPaper{300pt}{300pt}{Descargas/fondopapelviejo.jpg}
\chapter{Vogt(2015)의 첫번째 리뷰}
\section{Model setting}
\dfn 
$\{X_{t,T}:t=1,\dots,T\}_{T=1}^{\infty}$가 locally stationary라는 것은 모든 $u \in [0,1]$에 대하여 아래식을 만족하는 strictly stationary process $\{X_t(u):t \in \mathbb{Z}\}$를 찾을 수 있다는 것이다.
\[
\|X_{t,T}-X_t(u)\|\leq \left(\left| \frac{t}{T}-u \right|+\frac{1}{T} \right)U_{t,T}(u)~~a.s.
\]

\ck 이때 $u$는 rescaled time 이다. 

\note 보통 $u=\frac{t}{T}$로 정의하는데 이 정의는 모호한 편이다. 왜냐하면 $T=\infty$인 경우를 따지기 힘들기 때문임.

\ck 여기에서 $t$가 고정되었을때 $X_{t,T}$는 d차원 r.v.이고 따라서 $\| \cdot \|$는 $\mathbb{R}^d$에서의 norm이다. 

\ck 또한 $\{U_{t,T}:t=1,\dots,T\}_{T=1}^{\infty}$는 $E[U_{t,T}^{\rho}(u)] \leq C < \infty$를 만족하는 어떠한 양의 확률변수이다. 

\dfn 
$\lambda_{t,T}$는 $E(X_{t,T})$혹은 $V(X_{t,T})$와 같이 시간에 따라서 변화하는 locally stationary process $X_{t,T}$ 의 특징이다. 

\ck 여기에서 $\lambda_{t,T}$는 아래와 같은 성질을 만족한다. 

\one $\lambda_{t,T}$는 $\{X_{t,T}\}$에서 어떠한 measurable함수 $f:\mathbb{R}^d \rightarrow \mathbb{R}$를 취한것의 평균으로 표현된다. 즉 $$\lambda_{t,T}=E(f(X_{t,T}))$$이다. 여기에서 $f$들의 집합을 $\cal F$라고 표현한다. 	

\ex $X_{t,T}=(X_{t,T,1},\dots,X_{t,T,d})'$라고 하자. $\lambda_{t,T}=cov(X_{t,T,i},X_{t,T,j})$이면 ${\cal F}:=\{f_{ij}:1\leq i \leq j \leq d\} ~~ where~f_{ij}(x)=x_ix_j$

\two $\lambda_{t,T}=E(f(X_{t,T}))$는 아래를 만족한다. $$\sup_{f\in {\cal F}} \left|E[f(X_{t,T})]-E[f(X_t(u))]\right| \leq C \left(\left| \frac{1}{T}-u \right|+\frac{1}{T} \right).$$ 즉 $\lambda_{t,T}$가 locally하게는 $\lambda(u)$에 수렴한다는 의미이다. 여기에서 $\lambda(u)=E(f(X_t(u)))$

\rdash

\assume 우리는 $\lambda(u)$가 $[0,u_0]$에서는 변화하지 않고, $u_0$이후에서만 변화한다고 가정할 것이다. 

\goal 그리고 우리의 목적은 $u_0$를 찾는것, 즉 $\lambda(u)$가 시간에 따라 변화하기 시작하는 시점을 찾는 것이다. 

% \begin{figure}[t]
% \centering
% \includegraphics[width=1.0\linewidth]{cphist.pdf}
% \caption{Histograms of detected points by the proposed method over various $\tau$'s.}
% \label{fig:cphist}
% \end{figure}

\dash

\section{A measure of time-variation}

\dfn 
$D(u,v,f)$는 아래와 같이 정의된다. 
\[
D(u,v,f)=\int_0^v E(f(X_t(w)))dw-\left(\frac{v}{u}\right) \int_0^u E(f(X_t(w)))dw
\]

\ck 
따라서 고정된 $f$와 $u$에 대하여 
\[
\sup_{v\in[0,u]}|D(u,v,f)
\]
의 값이 0이라면, $E(f(X_t(w)))$는 구간 $[0,u]$에서 변화가 없다는 의미가 된다. 

\rdash

\dfn 
함수 ${\cal D}:[0,1] \rightarrow \mathbb{R}_{\geq 0}$을 아래와 같이 정의하자. 	
$${\cal D}(u)=\sup_{f \in {\cal F}}\sup_{v \in [0,u]} |D(u,v,f)|.$$

\ck 만약에 $X_{t,T}$가 strong stationary process라고 하면, 어떠한 함수 $f$에 대하여서도 
\[
\sup_{v \in [0,u]}|D(u,v,f)|=0.
\]
이다. 따라서 이때는 ${\cal D}(u)=0$이 된다. 

\rdash 

\ck 여기에서 $\int_0^vE[f(X_t(w))]dw$는 아래와 같이 추정할 수 있다. 
$$\frac{1}{T}\sum_{t=1}^{[vT]}f(X_{t,T}).$$
이때 앞에 $\frac{1}{T}$를 곱해준 이유는 시간축이 $[0,1]$로 scaling되어 있기 때문이다. 

\ck 따라서 $\hat{D}_T(u,v,f)$와 $\hat{\cal D}_T(u)$는 각각 아래와 같이 추정하는 것이 바람직 하다. 
$$\hat{D}_T(u,v,f)=\frac{1}{T}\sum_{t=1}^{[vT]}f(X_{t,T})-\left(\frac{v}{u}\right) \frac{1}{T} \sum_{t=1}^{[uT]}f(X_{t,T}).$$
$$\hat{\cal D}_T(u)=\sup_{f \in {\cal F}}\sup_{v \in [0,u]} |\hat{D}(u,v,f)|.$$

\section{Estimating the gradual change point $u_0$.}

\ck 일반적으로 
$$\hat{\cal D}_T(u)=\sup_{f \in {\cal F}}\sup_{v \in [0,u]} |\hat{D}(u,v,f)|.$$
는 $u\leq u_0$에서는 0에 가까운 값을 가지고 $u>u_0$에서는 어떠한 bounded된 값을 가지게 된다. 즉 $\hat{\cal D}_T(u)$는 아래와 같은 느낌이다. 
$$\hat{\cal D}_T(u)=\begin{cases}
\overset{p}{\longrightarrow} 0 ,& u\leq u_0, \\ 
\leq C < \infty, & u>u_0.
\end{cases}$$

\ck 또한 $\sqrt{T}\hat{\cal D}_T(u)$는 대충 아래와 같은 느낌을 가지게 된다. 
$$\sqrt{T}\hat{\cal D}_T(u)=\begin{cases}
\overset{d}{\longrightarrow} {\cal H}(u) ,& u\leq u_0, \\ 
\overset{p}{\longrightarrow} \infty, & u>u_0.
\end{cases}$$

\ck $\sqrt{T}\hat{\cal D}_T(u)$가 어떠한 분포를 가지지 않고 $\infty$값을 가지게 되는 시점을 찾기 위해서 아래와 같은 성질을 만족하는 $\tau_T$를 도입하자. 
$$\begin{cases}
P(\sqrt{T}\hat{\cal D}_T(u)\leq \tau_T) \rightarrow 1, &u\leq u_0 \\ 
P(\sqrt{T}\hat{\cal D}_T(u)\leq \tau_T) \rightarrow 0, &u > u_0
\end{cases}$$

\note 이를 위해서는 적당히 $\tau_T$를 ${\cal H}(u)$의 분포에서 나올수 있는 최대값으로 정의하는것이 바람직하다. 즉 $h_1,\dots,h_T$가 분포 ${\cal H}(u)$에서 나온 샘플이라고 한다면 대충 
$$\tau_T \approx \max\{h_1,\dots,h_T\}$$
와 같이 정의하는 것이 바람직하다. 이렇게 정의하면 자연스럽게 $\tau_T$는 $T$가 커질수록 무한대로 (천천히) 발산하게 됨을 이해할 수 있다.

\ck 위의 식을 다시 쓰면 
$$\begin{cases}
P(1(\sqrt{T}\hat{\cal D}_T(u)\leq \tau_T)=1) \rightarrow 1, &u\leq u_0 \\ 
P(1(\sqrt{T}\hat{\cal D}_T(u)\leq \tau_T)=1) \rightarrow 0, &u > u_0
\end{cases}$$
와 같이 쓸 수 있고 여기에서 $\hat{r}_T(u):=1(\sqrt{T}\hat{\cal D}_T(u)\leq \tau_T)$라고 정의하면 
$$\hat{r}_T(u) \overset{p}{\longrightarrow} \begin{cases}
1, & u\leq u_0 \\
0, & u>u_0
\end{cases}$$
와 같이 된다. 따라서 $\int_0^1\hat{\tau}_T(u)du$를 이용하여 $u_0$를 추정할 수 있다. 즉
$$\hat{u}_0(\tau_T)=\int_0^1\hat{\tau}_T(u)du$$
와 같이 쓸 수 있다. 

\dash 

\section{Asymptotic properties.}
\promise
$\ell_{\infty}(S)$를 supremum norm이 bound된 $f:S\rightarrow \mathbb{R}$들의 집합으로 정의하자. 

\promise 아래를 가정하자. 
$$\frac{{\cal D}(u)}{(u-u_0)^{\kappa}} \longrightarrow c_{\kappa} >0~~as~ u\downarrow u_0$$
여기에서 $\kappa$는 자연수이고 $c_{\kappa}$는 양수이다.

\ck 여기에서 큰 수의 $\kappa$는 $\cal D$가 $u_0$시점 이후로 좀 더 천천히 $0$으로부터 멀어짐을 의미한다. 즉 $\kappa$는 $\cal D$의 smoothness의 정도를 알려주는 파라메터이다. 

\note $u>u_0$에서 $u-u_0$는 1보다 작은 양수이므로, $\kappa$가 클수록 $(u-u_0)^{\kappa}$는 매우 천천히 변화한다.

\dash 

\subsection{Assumption.}

\ck 내용이없어서.. 

\dash 

\subsection{Weak convergence of the measure of time-variation.}

\ck 아래와 같은 통계량을 정의하자. 
$$\hat{H}_T(u,v,f)=\sqrt{T}\left(\hat{D}_T(u,v,f)-D(u,v,f)\right)$$ 
여기에서 $\hat{H}$, $\hat{D}_T$, $D$는 모두 공간 $\Delta \times {\cal F}$에서 $\mathbb{R}$로 가는 함수이다. 이때 $\Delta=\{(u,v)\in [0,1]^2:v\leq u\}$로 정의한다. 

\ck 추가적으로 $\hat{H}_T$는 $\ell_{\infty}(\Delta \times {\cal F})$의 원소라고 가정하자. 즉 $\hat{H}_T$의 sup이 bound되어있다고 가정하자.

\note 생각해보면 이 가정은 매우 특이하다. $X_{t,T}$가 $\Omega$에서 정의된 확률과정이듯이 $\hat{H}_T$는 $\Omega \times \Delta \times {\cal F}$에서 정의되는 일종의 확률변수로 볼 수 있는데, 이 확률변수의 최대값(=sup)이 bound되어 있다는 의미이기 때문이다.

\rdash 

\thm 적당한 가정하에서 통계량 $\hat{H}_T(u,v,f)$의 분포가 평균이 0이고 분산이 $$Cov(H(u,v,f),H(u',v',f'))$$인 Gaussian process $H$로 수렴한다. 즉, 
$$\hat{H}_T \overset{d}{\longrightarrow} H$$
이다. 

\ck $\hat{H}_T(u,v,f)$와 $\hat{H}_T(u',v',f')$는 모두 $X_{t,T}$로부터 유도된 확률변수이지만 엄연히 서로 다른 확률변수이다. 

\note $X$가 r.v.라고 할때 $Y_1=f_1(X)$, $Y_2=f_2(X)$과 같은 느낌

\ck 
따라서 두 확률변수의 분산을 구할수 있는데 이를 아래와 같이 정의한다. 
$$Cov(H(u,v,f),H(u',v',f')).$$

\rdash

\dfn 확률변수 $\hat{\cal D}_T(u)$, $\hat{\cal H}_T(u)$를 아래와 같이 정의하자. 
$$\hat{\cal D}_T(u)=\sup_{f\in {\cal F}}\sup_{v \in [0,u]} |\hat{D}_T(u,v,f)|.$$
$$\hat{\cal H}_T(u)=\sup_{f\in {\cal F}}\sup_{v \in [0,u]} |\hat{H}_T(u,v,f)|.$$
즉 $\hat{\cal D}_T(u)$는 고정된 $u$에 대하여 확률변수 $\hat{D}_T(u,v,f)$가 취할수 있는 값 중의 가장 큰 값을 의미하고, $\hat{\cal H}_T(u)$역시 그러하다. 

\note $\hat{\cal D}_T(u)$는 $\Omega \times [0,1]$에서 정의된 확률변수라고 볼 수 있고 $\hat{\cal H}_T(u)$역시 그러하다.

\note 만약에 $u=0.6$으로 이라고 한다면 $\hat{\cal D}_T(0.6)$은 아래와 같은 확률변수들 중에서 최대값이다.
$$\hat{D}_T(0.6,0.5,f_1)$$
$$\hat{D}_T(0.6,0.4,f_1)$$
$$\hat{D}_T(0.6,0.3,f_1)$$
$$\hat{D}_T(0.6,0.2,f_1)$$
$$\dots$$
$$\hat{D}_T(0.6,0.5,f_2)$$
$$\hat{D}_T(0.6,0.4,f_2)$$
$$\hat{D}_T(0.6,0.3,f_2)$$
$$\hat{D}_T(0.6,0.2,f_2)$$
$$\dots$$ 

\rdash 

\dfn 확률변수 $\hat{\mathbb D}_T(u)$, $\hat{\mathbb H}_T(u)$를 아래와 같이 정의하자. 
$$\hat{\mathbb D}_T(u)=\sup_{f\in {\cal F}}\sup_{0\leq w \leq v \leq u} |\hat{D}_T(v,w,f)|.$$
$$\hat{\mathbb H}_T(u)=\sup_{f\in {\cal F}}\sup_{0\leq w \leq v \leq u} |\hat{H}_T(v,w,f)|.$$

\note 만약에 $u=0.6$으로 이라고 한다면 $\hat{\mathbb D}_T(0.6)$은 아래와 같은 확률변수들 중에서 최대값이다.
	$$\hat{D}_T(0.6,0.5,f_1)$$
	$$\hat{D}_T(0.6,0.4,f_1)$$
	$$\hat{D}_T(0.6,0.3,f_1)$$
	$$\hat{D}_T(0.6,0.2,f_1)$$
	$$\dots$$
	$$\hat{D}_T(0.4,0.2,f_1)$$
	$$\hat{D}_T(0.4,0.1,f_1)$$
	$$\dots$$ 
	$$\hat{D}_T(0.3,0.2,f_2)$$
	$$\hat{D}_T(0.3,0.1,f_2)$$
	$$\dots$$ 

\note 따라서 자연스럽게 $\hat{\mathbb D}_T(u) \geq \hat{\cal D}_T(u)$임을 알 수 있다. 

\ck 아래와 같이 정의하기도 한다. 
$$\hat{\mathbb D}_T(u)=\sup_{v\in [0,u]} |\hat{\cal D}_T(v)|.$$
$$\hat{\mathbb H}_T(u)=\sup_{v\in [0,u]} |\hat{\cal H}_T(v)|.$$

\rdash 

\thm 적당한 조건하에서 $H_T$가 $H$로 분포수렴한다는 정리를 이용하면 쉽게 아래를 보일 수 있다.
$$ \sup_{f\in {\cal F}}\sup_{v \in [0,u]} |\hat{H}_T(u,v,f)| \overset{d}{\longrightarrow} \sup_{f\in {\cal F}}\sup_{v \in [0,u]} |{H}(u,v,f)|.$$
$$ \sup_{f\in {\cal F}}\sup_{0\leq w \leq v \leq u} |\hat{H}_T(v,w,f)| \overset{d}{\longrightarrow} \sup_{f\in {\cal F}}\sup_{0\leq w \leq v \leq u} |{H}(v,w,f)|.$$

\dash 
\subsection{Convergence of the estimator $\hat{u}_0(\tau_T)$.}

\ck
앞서서 $\hat{\cal D}$는 아래와 같은 느낌이라고 하였다. 
$$\hat{\cal D}_T(u)=\begin{cases}
\overset{p}{\longrightarrow} 0 ,& u\leq u_0, \\ 
\leq C < \infty, & u>u_0.
\end{cases}$$
그런데  
$$\begin{cases}
P(\sqrt{T}\hat{\cal D}_T(u)\leq \tau_T) \rightarrow 1, &u\leq u_0 \\ 
P(\sqrt{T}\hat{\cal D}_T(u)\leq \tau_T) \rightarrow 0, &u > u_0
\end{cases}$$
이므로 아래가 성립한다. 
$$\begin{cases}
P(\hat{\cal D}_T(u)\leq \frac{\tau_T}{\sqrt{T}}) \rightarrow 1, &u\leq u_0 \\ 
P(\hat{\cal D}_T(u)\leq \frac{\tau_T}{\sqrt{T}}) \rightarrow 0, &u > u_0
\end{cases}$$

\promise 이런 사실을 종합하면 $\tau_T$가 아래의 성질을 만족해야 함을 유추할 수 있다. 
\begin{enumerate}[(i)]
\item  $\tau_T \longrightarrow \infty$ 
\item  $\tau_T/\sqrt{T} \longrightarrow 0$ 
\end{enumerate}
따라서 앞으로는 특별한 언급이 없어도 $\tau_T$가 위의 두 성질을 만족한다는 것으로 생각하자. 

\rdash 

\dfn
$\gamma_T$를 아래와 같이 정의하자. 
$$\gamma_T=\left(\frac{\tau_T}{\sqrt{T}}\right)^{1 / \kappa}$$

\note 이때 $\kappa$는 앞에서 약속한것 처럼 다음식을 만족시켜야 한다. 
$$\frac{{\cal D}(u)}{(u-u_0)^{\kappa}} \longrightarrow c_{\kappa} >0~~as~ u\downarrow u_0$$

\ck $\left(\gamma_T\right)^{\kappa}$는 $T$가 커짐에 따라 $0$으로 수렴한다. 따라서 $\gamma_T$역시 $T$가 커짐에 따라서 0으로 수렴한다. 동일한 $\kappa$일경우 굳이 따지면 $\left(\gamma_T\right)^{\kappa}$이 $\gamma_T$보다 빠르게 수렴한다. 

\ck 
그런데 $P(\hat{\cal D}_T(u)\leq \frac{\tau_T}{\sqrt{T}}) \rightarrow 1$임을 생각하면 대충 $T$가 무한대로 감에 따라 ${\cal D}(u)\leq \left(\gamma_T\right)^{\kappa}$와 같은 식이 성립한다고 생각할 수 있다. 또한 
$$\frac{{\cal D}(u)}{(u-u_0)^{\kappa}} \longrightarrow c_{\kappa} >0~~as~ u\downarrow u_0$$
임을 감안하면, 아래식이 성립하는 느낌이 든다. 
$$\frac{\left(\gamma_T\right)^{\kappa}}{(u-u_0)^{\kappa}} \longrightarrow c_{\kappa} >0~~as~ u\downarrow u_0,~T\rightarrow \infty.$$
즉 $T \rightarrow \infty$일때 $(\gamma_T)^{\kappa}$가 $0$으로 가는 속도는 $u\downarrow u_0$일때 $(u-u_0)^{\kappa}$가 $0$으로 가는 속도와 비슷하다. 즉 $\gamma_T$는 적당하게 $u-u_0$가 $0$으로 가는 속도와 비슷하도록 $\kappa$와 $\tau_T$를 조절하여 얻을 수 있다. 기본적으로 $\gamma_T$의 값은 $\kappa$와 $\tau_T$에 따라서 아래와 같은 규칙으로 변화한다. 

\one $\kappa$가 커질수록, 즉 $\cal D$의 부드러움의 정도가 클수록 $\gamma_T$는 천천히 0으로 수렴한다. 

\two $\tau_T$가 천천히 무한대로 갈수록, 즉 $\sqrt{T}\hat{\cal D}_T$의 극한분포 $\cal H$에서 나오는 최대값을 좀더 타이트하게 잡을수록 $\gamma_T$는 빠르게 0으로 수렴한다. 

\thm 적당한 조건하에서 아래가 성립한다. 
$$\hat{u}_0(\tau_T)-u_0=O_p(\gamma_T).$$

\ck $\gamma_T$는 기본적으로 $\cal D$의 부드러움의 정도가 클수록 천천히 0으로 수렴하기 때문에 $\hat{u}_0(\tau_T)$이 보다 정확한 추정량이 되기 위해서는 $u_0$에서 파라메터가 서서히 변화하는 모델보다 파라메터가 급격하게 변화하는 모델이 유리함을 알 수 있다. 

\ck 또한 $\tau_T$가 좀 더 천천히 발산할수록 $\gamma_T$를 보다 빠르게 0으로 수렴시킬수 있기 때문에 $\sqrt{T}\hat{\cal D}_T$의 극한분포 $\cal H$에서 나오는 최대값을 좀더 타이트 하게 잡을수록 $u_0$를 정확하게 추정할 수 있음을 알 수 있다. 

\dash 
\subsection{Some mean squared error considerations.}


\dash 

\subsection{Choice of the threshold level $\tau_T$}


\promise $q_{\alpha}(u)$를 $\mathbb{H}(u)$의 $1-\alpha$-분위수라고 약속하자. 

\promise 모든 $u\in [0,1]$에 대하여 $\mathbb{H}(u)$의 분포를 알고있다고 약속하자. 

\note 실제로는 분포를 몰라서 적당한 방법을 통하여 추론을 해야한다. 이런 방법은 다음장(Implementation)에서 자세히 다룸.)

\ck 참고로 $u<u_0$인곳에서 $\mathbb{H}(u)=\sqrt{T}\mathbb{D}(u)$임을 기억하자. 
\ck 또한 $\mathbb{H}(u)$의 분포는 고정된 $u$에 대해서 정규확률과정 $H(u,v,f)$에서 나오는 확률변수들의 최대값들의 분포를 의미한다. 

\rdash 

\subsubsection{Preliminary choice of $\tau_T$}
\ck $H_0$를 ``변화점이 없다.''로 설정하자. 우선 우리의 첫번째 목표는 귀무가설하에서 $$P(\hat{u_0}(\tau_T)<1)=\alpha$$가 되도록 하는 어떠한 $\tau_T$를 구하는 것이다. 
\ck 우선 $\tau_T$가 $T$에 depend하지 않는 어떠한 상수 $\tau$라고 하자. 즉 모든 $T$에 대하여 $\tau_T=\tau$이다. 
\ck 아래식이 성립한다. 
$$P(\hat{u}_0(\tau)<1)\leq P(\sqrt{T}\hat{\cal D}_T(u)>\tau,~\textup{for some}~u\in[0,1])=P(\sqrt{T}\mathbb{D}_T(1)>\tau)$$
그런데 changing point가 없다고 하였으므로 
$$P(\sqrt{T}\mathbb{D}_T(1)>\tau) \longrightarrow P(\mathbb{H}(1)>\tau)$$
가 된다. $\mathbb{H}(1)$의 분포를 완전히 알고 있다고 가정하였으므로 그것의 $1-\alpha$-분위수 $q_{\alpha}(1)$를 정확하게 알 수 있다. 이것을 $\tau$로 잡자. 즉 $\tau=q_{\alpha}(1)$로 잡자. 그러면, 
$$P(\hat{u}_0(\tau)<1)\leq \alpha +o(1)$$
이 된다. 이때의 $\tau$를 $\tau_{\alpha}^{\circ}$이라고 하자. 즉 $\tau_{\alpha}^{\circ}=q_{\alpha}(1)$. 그러면 위의식을 아래와 같이 다시 쓸 수 있다. 
$$P(\hat{u}_0(\tau_{\alpha}^{\circ})<1)\leq \alpha +o(1)$$
\ck 하지만 $\tau_{\alpha}^{\circ}$는 앞에서 가정한것 처럼 $T$가 커짐에 따라서 (천천히) 무한대로 발산하는 수가 아니라 $T$에 영향을 받지 않는 어떠한 상수이다. 따라서 $\hat{u}_0(\tau_{\alpha}^{\circ})$는 consistent estimator가 될 수 없다. $\hat{u}_0(\tau_{\alpha}^{\circ})$를 consitent estimator가 되게 하려면 $\tau_{\alpha}^{\circ}$를 무한대로 천천히 발산시켜야 한다. 그 방법은 $T$가 커짐에 따라서 $\alpha$를 0으로 보내야 한다. 즉 아래와 같은 수열을 정의해야 한다. 
$$\alpha_T \rightarrow 0,~as~T\rightarrow \infty$$
여기에서 $\alpha_T$를 충분히 천천히 0으로 보내야 함을 이해해야 한다. $\alpha_T$를 빠르게 0으로 보내면 어떤일이 생길까? 그렇다면 $\tau_T$가 $\sqrt{T}$보다 빠르게 무한대로 발산할 수도 있을 것이고 이것은 적절한 $\tau_T$에 대한 선택이 아니다. 

\ck 얼마나 천천히 $\alpha_T$를 0으로 보내야 하는 것일까? 만족시켜야 할것은 
$$\frac{\tau_{\alpha_T}^{\circ}}{T} \rightarrow 0$$
이어야 한다는 것이다. 그런데 이것은 적당한 $c$와 $r$에 대하여 $\alpha_T\leq cT^{-r}$를 만족하는 $\alpha_T$만 잡으면 된다고 논문에 나와있다. 

\ck 실제로 $T$는 고정되어 있으므로, 천천히 수렴하는 수열 $\{\alpha_T\}$를 잡는다느니 하는 말은 다 이론용이다. 실제로는 하나의 $\alpha_T$만 정해주면 된다. 

\rdash 

\subsubsection{Refined choice of $\tau_T$}
\ck 귀무가설을 살짝 수정하여 $u_0$이전까지는 변화점이 없다고 하자. 
\ck 편의상 $u_0$를 알고 있다고 생각하자.  
\ck 아래식이 성립한다. 
$$P(\hat{u}_0(\tau)<u_0)\leq P(\sqrt{T}\hat{\cal D}_T(u)>\tau,~\textup{for some}~u\in[0,u_0])=P(\sqrt{T}\mathbb{D}_T(u_0)>\tau)$$
그런데 $u_0$까진 changing point가 없다고 하였으므로 
$$P(\sqrt{T}\mathbb{D}_T(u_0)>\tau) \longrightarrow P(\mathbb{H}(u_0)>\tau)$$
가 된다. $\mathbb{H}(u_0)$의 분포를 완전히 알고 있다고 가정하였으므로 그것의 $1-\alpha$분위수 $q_{\alpha}(u_0)$를 정확하게 알 수 있다. 이것을 $\tau_{\alpha}$로 잡자. 그러면 아래가 성립한다. 
$$P(\hat{u}_0(\tau_{\alpha})<u_0)\leq \alpha +o(1)$$


\ck 추가적으로 아래도 성립함을 논문에서 증명하였다. 
$$P(\hat{u}_0(\tau_{\alpha})>u_0+C\gamma_T)=o(1)$$

\thm 
적당한 가정하에서 아래식들이 성립한다.
$$P(\hat{u}_0(\tau_{\alpha})<u_0)\leq \alpha +o(1)$$
$$P(\hat{u}_0(\tau_{\alpha})>u_0+C\gamma_T)=o(1)$$
여기에서 $C>0$이고, $\gamma_T=\left(\frac{\tau_T}{\sqrt{T}}\right)^{1 / \kappa}$이다. 

\ck 실제로는 $u_0$를 모르기 때문에 $\tau_{\alpha}$, 혹은 $\{\tau_{\alpha_T}\}$를 잡을 수 없다. 따라서 아래의 과정을 통하여 $u_0$의 추정치혹은 추정치열을 구하고 그것을 이용하여 $\tau_{\alpha}$혹은 $\{\tau_{\alpha_T}\}$를 잡아야 한다. 
\begin{enumerate}[(i)]
	\item 0으로 천천히 수렴하는 적당한 수열 $\{\alpha_T\}$를 잡는다. 여기에서 $\alpha_T$는 적당한 $c$와 $r$에 대하여 $\alpha_T\leq cT^{-r}$를 만족하기만 하면 된다. 
	\item $\mathbb{H}(1)$의 분포를 알고 있으므로 $\mathbb{H}(1)$의 $1-\alpha_T$분위수에 해당하는 값들을 각각 구하고 이를 $\{\tau_{\alpha_T}^{\circ} \}$로 정의한다. \textcolor{red}{(1차선택!!)}
	\item 각각의 $\{\tau_{\alpha_T}^{\circ} \}$에 대하여 $\hat{\mu}_0(\tau_{\alpha_T}^{\circ})$를 계산한다. 아래식을 써야한다. 
	$$\hat{\mu}_0(\tau_{\alpha_T}^{\circ})=\int_0^1 1(\sqrt{T}\hat{\cal D}_T(u)\leq \tau_{\alpha_T}^{\circ}) du.$$
	여기에서 
	$$\hat{\cal D}_T(u)=\sup_{f \in {\cal F}}\sup_{v \in [0,u]} \left|\frac{1}{T}\sum_{t=1}^{[vT]}f(X_{t,T})-\left(\frac{v}{u}\right) \frac{1}{T} \sum_{t=1}^{[uT]}f(X_{t,T})\right|$$
	이다. 따라서 이 과정을 계산할때 주어진 샘플들, 즉 $\{X_{t,T}\}$이 사용된다. 
	\item $\hat{\mu}_0(\tau_{\alpha_T}^{\circ})$를 값을 넣어서 $\mathbb{H}(\hat{\mu}_0(\tau_{\alpha_T}^{\circ}))$의 분포를 구하고 또 그것의 $1-\alpha_T$분위수에 해당하는 값들을 각각 구하고 이를 $\{\hat{\tau}_{\alpha_T}^{\circ} \}$로 정의한다. 이것이 $\{\tau_T\}$이다. \textcolor{red}{(2차선택!!)}
\end{enumerate}

\thm
$\tau_{\alpha}$대신에 $\hat{\tau}_{\alpha}$를 넣어도 위의 정리가 성립한다. 즉 적당한 가정하에서 아래식들이 성립한다.
$$P(\hat{u}_0(\hat{\tau}_{\alpha})<u_0)\leq \alpha +o(1)$$
$$P(\hat{u}_0(\hat{\tau}_{\alpha})>u_0+C\gamma_T)=o(1)$$
여기에서 $C>0$이고, $\gamma_T=\left(\frac{\tau_T}{\sqrt{T}}\right)^{1 / \kappa}$이다. 

\chapter{Vogt(2015)의 두번째 리뷰}
\section{Introduction}
\ck goal
\section{Model setting} 

\ck 시계열 $\{X_t\}$ 가 locally stationary process 라고 하자. 

\ck 우리는 $\{X_t\}$ 에서 생성된 하나의 realization $\{x_t\}$ 를 관측했다고 하자. 

\ck Let $\lambda_t$ be some time-varying feature of $X_t$. 
\ex $\lambda_t=E(X_t)$. 
\ex $\lambda_t=V(X_t)$. 

\note 만약에 $\{X_t\}$가 정상이라면 $\lambda_t$는 항상 상수이어야 할것이다. 

\ck $\lambda_t$는 구체적으로 (1) 시계열 $\{X_t\}$ 에서 (2) 임의의 measurable function $f \in {\cal F}$를 취한 후 (3) 그것의 평균으로 정의된다. 즉 아래와 같은 형태를 가져야한다. 
\[
\lambda_t = E(f(X_t)), \quad f \in {\cal F}
\]

\ex $\lambda_t=E(X_t)$인 경우 ${\cal F}=\{\mbox{id}\}$. 

\rdash 

\ck 관측한 시계열이 univariate timeseries 이지만 $\lambda_t$가 벡터일수도 있다. (이때는 univariate를 관측했지만 multivariate를 관측했다고 가정한다) 아래는 그 예제이다. 

\ex $\{Y_t\}$가 univariate timeseries라고 하자. 우리는 아래에 관심이 있다고 하자. 
\[
\bld{\gamma}=
\begin{bmatrix}
\gamma_0 \\ 
\gamma_1 \\ 
\dots \\
\gamma_p
\end{bmatrix}
\]
이럴때는 
\[
\bld{\lambda}_t=\bld{\gamma}
\]
로 설정한다. 여기에서 $\gamma_{\ell}=Cov(X_t,X_{t-\ell})$을 의미함. 

\begin{thebibliography}{9} %% 9의 의미는 "[1] 해석학입문 (4판, William R. Wade)"가 얼마나 왼쪽으로 들여쓰기 되는지에 대한 척도. 
\bibitem{ref} Vogt, M., \& Dette, H. (2015). Detecting gradual changes in locally stationary processes. The Annals of Statistics, 43(2), 713-740.
\end{thebibliography}

\end{document}

